%File: ~/OOP/material/section/ElasticSection2D.tex
%What: "@(#) ElasticSection2D.tex, revA"

\noindent {\bf Files}   \\
\#include $<\tilde{ }$/material/section/ElasticSection2D.h$>$  


\noindent {\bf Class Declaration}  \\
class ElasticSection2D : public SectionForceDeformation 


\noindent {\bf Class Hierarchy} \\
TaggedObject 

MovableObject 

\indent\indent Material \\
\indent\indent\indent SectionForceDeformation \\
\indent\indent\indent\indent {\bf ElasticSection2D} \\

\noindent {\bf Description}  \\
\indent ElasticSection2D provides the implementation of a
section which exhibits uncoupled elastic behavior in axial, moment,
and shear response. \\

\noindent {\bf Class Interface} \\
// Constructor 

\indent {\em ElasticSection2D (int tag, double E, double A, double I,
double G, double alpha = 5/6.0);} \\ 
{\em ElasticSection2D ();} \\ 

// Destructor 

{\em $\tilde{ }$ElasticSection2D ();} \\ 

// Public Methods 

{\em int setTrialSectionDeformation (const Vector \&def); } 

{\em const Vector \&getSectionDeformation (void); } 

{\em const Vector \&getStressResultant (void); } 

{\em const Vector \&getPrevStressResultant (void); } 

{\em const Matrix \&getSectionTangent (void); } 

{\em const Matrix \&getPrevSectionTangent (void); } 

{\em const Matrix \&getSectionFlexibility (void); } 

{\em const Matrix \&getPrevSectionFlexibility (void); } 

{\em int commitState (void); } 

{\em int revertToLastCommit (void); } 

{\em int revertToStart (void); } 

{\em SectionForceDeformation *getCopy (void); } 

{\em const ID \&getType (void); } 

{\em int getOrder (void); } \\ 

// Public Methods for Output

{\em int sendSelf (int commitTag, Channel \&theChannel); }

\indent {\em int recvSelf (int commitTag, Channel \&theChannel, 
FEM\_ObjectBroker \&theBroker); }\\
{\em void Print (OPS_Stream \&s, int flag = 0);} 


\noindent {\bf Constructor}  \\
\indent {\em ElasticSection2D (int tag, double E, double I, double A,
double G, double alpha);}  \\
To construct an ElasticSection2D with an integer identifier {\em
tag}, an elastic modulus of $E$, a second moment of area $I$,
a section area of $A$, an elastic shear modulus of $G$, and a shape factor
of $alpha$. The integers {\em tag} and MAT\_TAG\_ElasticSection2D, defined in
$<$classTags.h$>$, are passed to the SectionForceDeformation
class constructor. \\

{\em ElasticSection2D ();} 

Constructs a blank ElasticSection2D object. \\

\noindent {\bf Destructor} \\
\indent {\em $\tilde{}$ElasticSection2D ();} \\ 
Does nothing. \\

\noindent {\bf Public Methods} \\
{\em int setTrialSectionDeformation (const Vector \&def); }  

Sets the value of the section deformation vector, $\esec$ to be {\em
def}. The section deformation vector, $\esec$, is defined by

\begin{equation}
\esec := \left[
   \begin{array}{c}
       \epsilon_a \\
       \kappa_z   \\
       \gamma_y
   \end{array} 
 \right]
\end{equation}

where $\epsilon_a$ is the axial strain, $\kappa_z$ is the curvature about
the local z-axis, and $\gamma_y$ is the shear strain along the local y-axis.
Returns $0$. \\

{\em const Vector \&getSectionDeformation (void); } 

Returns the current value of the section deformation vector, $\esec$. \\

{\em const Vector \&getStressResultant (void); } 

Returns the section stress resultants, $\ssec$, the product of the 
section stiffness matrix, $\ksec$, and the section deformation 
vector, $\esec$,

\begin{equation}
\ssec = \ksec \esec = \left[
   \begin{array}{c}
       P \\
       M_z   \\
       V_y
   \end{array} 
 \right]
\end{equation}

where $P$ is the axial force, $M_z$ is the bending moment about the
local z-axis, and $V_y$ is the shear force along the local y-axis. \\

{\em const Vector \&getPrevStressResultant (void); } 

Returns the value of $\ssec$ calculated at the previous state determination. \\

{\em const Matrix \&getSectionTangent (void); } 

Returns the section stiffness matrix, $\ksec$, where 

\begin{equation}
\ksec = \left[
   \begin{array}{ccc}
       EA &  0 &  0 \\
        0 & EI &  0 \\
        0 &  0 & \alpha GA
   \end{array} 
 \right]
\end{equation} \\

{\em const Matrix \&getPrevSectionTangent (void); } 

Returns the section stiffness matrix, $\ksec$. \\

{\em const Matrix \&getSectionFlexibility (void); } 

Overrides the base class implementation and returns the section
flexibility matrix, $\fsec$, where

\begin{equation}
\fsec = \left[
   \begin{array}{ccc}
       \frac{1}{EA} &            0 &                  0  \\
                  0 & \frac{1}{EI} &                  0  \\
                  0 &            0 & \frac{1}{\alpha GA}
   \end{array} 
 \right]
\end{equation} \\

{\em const Matrix \&getPrevSectionFlexibility (void); } 

Overrides the base class implementation and returns the section
flexibility matrix, $\fsec$. \\

{\em int commitState (void); } 

Returns $0$. \\

{\em int revertToLastCommit (void); } 

Returns $0$. \\

{\em int revertToStart (void); } 

Returns $0$. \\

{\em SectionForceDeformation *getCopy (void); } 

Returns a pointer to a new ElasticSection2D object, constructed
using the same values of {\em tag}, $E$, $A$, $I$, $G$, and $\alpha.  
It is up to the caller to ensure that the destructor is invoked. \\

{\em const ID \&getType (void); } 

Returns the section ID code that indicates the ordering of
section response quantities. For this section, axial response is the
first quantity, bending about the local z-axis is the second, and
shear along the local y-axis is third. \\

\begin{equation}
code := \left[
   \begin{array}{c}
       2 \\
       1 \\
       3 
   \end{array} 
 \right]
\end{equation} \\

{\em int getOrder (void); } 

Returns 3. \\

{\em    int sendSelf(int commitTag, Channel \&theChannel); }

Creates a Vector of size $6$ into which it places  {\em tag},
$E$, $A$, $I$, $G$, and $\alpha$. Invokes {\em sendVector()} on  {\em theChannel} using the
ElasticSection2D objects {\em dbTag}, the integer {\em commitTag} and
the Vector as arguments. Returns $0$ if successful, a warning message
and a negative number are returned if the Channel object fails to send
the Vector. \\

\indent {\em    int recvSelf(int commitTag, Channel \&theChannel, 
		 FEM\_ObjectBroker \&theBroker); }\\
Creates a Vector of size $6$. Invokes {\em recvVector()} on  {\em
theChannel} using the ElasticSection2D objects {\em dbTag}, the integer
{\em commitTag} and the Vector as arguments. Using the data in the
Vector to set its {\em tag}, $E$, $A$, $I$, $G$, and $\alpha$. Returns $0$ if successful, a
warning message is printed and a negative number is returned if the Channel 
object fails to receive the Vector. \\

{\em    void Print(OPS_Stream \&s, int flag =0);} 

Prints to the stream {\em s} the objects {\em tag}, $E$, $A$, $I$, $G$, and
$\alpha$ values. \\



