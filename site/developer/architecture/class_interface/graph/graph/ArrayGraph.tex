%File: ~/OOP/graph/graph/ArrayGraph.tex
%What: "@(#) ArrayGraph.tex, revA"

UNDER CONSTRUCTION - removeVertex \& printSpecial NEED TO BE ADDED. \\

\noindent {\bf Files}   \\
\#include $<\tilde{ }$/graph/graph/ArrayGraph.h$>$  


\noindent {\bf Class Decleration}  \\
class ArrayGraph: public Graph 


\noindent {\bf Class Hierarchy} \\
Graph 

\indent\indent {\bf ArrayGraph} \\

\noindent {\bf Description}  \\
\indent ArrayGraph is a subtype of Graph. The vertices for this type
of graph are held in a simple array data structure whose initial size
is specified at construction. This size can increase if
needed. The array storage scheme is more efficient than a List storage
scheme in terms of accessing the vertices; in very large problems
where memory is limited this type of scheme may have problems getting
enough contiguous meory in which case a List might be a better
choice. {\bf There is a question as to whether or not the public
methods should be declared as virtual. Good OOP programming would have
all methods declared as virtual, however as subclasses cannot gain
access to the private member data there does not seem to be much point
in declaring them, except for the destructor, virtual in this
instance.} \\ 


\noindent {\bf Class Interface}  \\
\indent\indent // Constructors  \\
\indent\indent {\em ArrayGraph(int arraySize);}  \\ \\
\indent\indent // Destructor  \\
\indent\indent {\em virtual~$\tilde{}$ArrayGraph();}  \\ \\
\indent\indent // Public Methods   \\
\indent\indent {\em virtual int addVertex(Vertex *vertexPtr);}  \\
\indent\indent {\em virtual int addEdge(int vertexTag, int otherVertexTag); } \\
\indent\indent {\em virtual Vertex *getVertexPtr(int vertexTag);} \\
\indent\indent {\em virtual VertexIter \&getVertices(void);} \\
\indent\indent {\em virtual int getNumVertex(void) const;} \\
\indent\indent {\em virtual int getNumEdge(void) const;} \\
\indent\indent {\em virtual void Print(OPS_Stream \&s) const =0;} \\ \\
\indent\indent // Protected Methods \\
\indent\indent {\em virtual int getArraySize(void) const;} \\


\noindent {\bf Constructors}  \\
{\em ArrayGraph(int arraySize);}  

To construct an empty ArrayGraph. Creates a Vertex ** array, {\em
theVertices} of size {\em arraySize} and sets the number of vertices,
{\em numVertex}, and number of edges {\em numEdge} to $0$. If it fails
to get an array of appropriate size it sets its {\em arraySize} to
$0$; subclasses can check if successfull construction by invoking the
protected member function {\em getArraySize()}. \\

\noindent {\bf Destructor}  \\
{\em virtual~$\tilde{}$ArrayGraph();}  

Goes through {\em theVertices} and anywhere it finds a non-zero pointer,
invokes the vertex destructor on that pointer. It then invokes the
destructor on theVertices array. \\


\noindent {\bf Public Methods }  \\
{\em virtual int addVertex(Vertex *vertexPtr);}  

Method to add a vertex to the graph. If the adjacency list
of the vertex is not empty the graph will first check to see all
vertices in the the the vertices adjacency list exist in the graph
before the vertex is added. It then checks if it neeeds a new array
and if so creates one, i.e. if the {\em arraySize} $=$ {\em
numVertex} it creates a new array, whose size is double the original
and copies the pointers to the vertices, before invoking {\em
delete()} on the old array. It now tries to add the vertex in the
array at location {\em vertexTag}. If this fails it adds at the first
empty location it comes to. Returns a 0 if successfull addition, a
$-1$ otherwise and a message to opserr explaining the problem. \\ 

{\em virtual int addEdge(int vertexTag, int otherVertexTag); } \\
Causes the Graph to add an edge {\em (vertexTag,otherVertexTag)} to
the Graph. A check is first made to see if vertices with tags given by
{\em vertexTag} and {\em otherVertexTag} exist in the graph. If they
do not exist a $-1$ is returned, otherwise the method invokes {\em
addEdge()} on each of the corresponding vertices in the 
graph. Returns $0$ or a positive integer if sucessfull (positive if
edge has already been added), a negative number if not.\\ 


{\em virtual Vertex *getVertexPtr(int vertexTag);} \\
A method which returns a pointer to the vertex whose tag is given by {\em
vertexTag}. The method first looks at location {\em vertexTag} for the
vertex, otherwise it must search through the array until it finds the
vertex it is looking for. If no such vertex exists in the graph $0$ is
returned.\\ 

{\em virtual VertexIter \&getVertices(void);} \\
A method which first invokes {\em reset()} on the graphs ArrayVertexIter
and then returns a reference to this iter.\\

{\em virtual int getNumVertex(void) const;} \\
A method to return the number of vertices in the graph, returns numVertex. \\

{\em virtual int getNumEdge(void) const;} \\
A method to return the number of edges in the graph, returns numEdge. \\


{\em virtual void Print(OPS_Stream \&s) const =0;} \\
A method to print the graph. It first prints out numVertex and numEdge
and then on each newline prints the vertexTag and the edges for that
vertex. It does this by going through theVertices array and invoking
{\em Print()}  on each non-zero pointer.\\

\noindent {\bf Protected Methods }  \\
{\em virtual int getArraySize(void) const;} 

A method to return the size of the graphs array. \\





