%File: ~/OOP/graph/graph/Metis.tex
%What: "@(#) Metis.tex, revA"

\noindent {\bf Files}   \\
\#include $<\tilde{ }$/graph/partitioner/Metis.h$>$  


\noindent {\bf Class Declaration}  \\
class GraphPartitioner:  


\noindent {\bf Class Hierarchy} \\
GraphPartitioner 

\indent\indent {\bf Metis} \\

\noindent {\bf Description}  \\
\indent Metis is a GraphPartitioner. The Metis graph partitioner calls
procedures defined in the METIS library to partition the graph. METIS
is currently being developed by G.~Karypis and V.~Kumar at the
University of Minnesota. At the present time the Graph to be
partitioned MUST have the vertices labeled $0$ through $numVertex-1$. \\

The METIS library uses two integer arrays to represent the graph, {\em
xadj} and {\em adjncy}. $xadj(i)$ stores the location in {\em adjncy}
of the start of the $i$'th Vertices adjacent Vertices. {\em adjncy}
contains the tags of all the adjacent vertices. For example, the graph
which is represented by the following matrix $A$:


$$ A =
\left[
\begin{array}{ccccc}
1 & 0 & 1 & 1 & 0  \\
1 & 1 & 0 & 0 & 0  \\
0 & 1 & 1 & 0 & 0 \\
0 & 0 & 0 & 1 & 1 \\
1 & 1 & 0 & 0 & 1
\end{array}
\right] 

$$

$$


is represented by:


$$

$$

xadj =
\left[
\begin{array}{cccccccccccccc}
0 & 2 & 3 & 4 & 5 & 7
\end{array}
\right] 

$$

$$


and


$$

$$

adjncy =
\left[
\begin{array}{cccccccccccccc}
2 & 3 & 0 & 1 & 4 & 0 & 1
\end{array}
\right] 

$$

$$


\noindent note that there is no space allocated for the diagonal
components. \\

\noindent {\bf Class Interface}  \\
\indent\indent // Constructors  \\
\indent\indent {\em Metis();}  \\ \\
\indent\indent {\em Metis(int pType, int mType, int coarsenTo, int 
rType, int ipType);}  \\ \\
\indent\indent // Destructor  \\
\indent\indent {\em virtual~$\tilde{}$Metis();}  \\ \\
\indent\indent // Public Methods  \\
\indent\indent {\em virtual int partition(Graph \&theGraph, int numPart) =0;} \\
\indent\indent {\em bool setDefaultOptions(void);}\\
\indent\indent {\em bool setOptions(int pType, int mType, int coarsenTo, int 
rTypem, int ipType);}  \\ \\
\indent\indent // Private Method  \\
\indent\indent {\em bool checkOptions(void) const;} \\


\noindent {\bf Constructors}  \\
{\em Metis();}  

To construct a Metis object which will use the default settings when
partitioning. \\ 

\indent {\em Metis(int pType, int mType, int coarsenTo, int 
rTypem, int ipType);}  \\
To construct a Metis object which will use the setting passed into the
constructor as options to metis's {\em PMETIS()} routine. {\em
checkOptions()} is invoked to ensure the settings are valid. \\

\noindent {\bf Destructor}  \\
{\em virtual~$\tilde{}$Metis();}  


\noindent {\bf Public Methods }  \\
{\em virtual int partition(Graph \&theGraph, int numPart) =0;} 

This is the method invoked to partition the graph into {\em numPart}
partitions. On completion of the routine each vertex will be assigned
a color $1$ through {\em numPart}, the color assigned indicating the
partition to which the vertex belongs. 

To partition a number of integer arrays are created, {\em options[5]},
{\em partition[numVertex+1]}, {\em xadj[numVertex+1]} and {\em
adjncy[2*numEdge]} (CURRENTLY ASSUMING GRAPH IS SYMMETRIC - THIS MAY
CHANGE \& xadj and partition 1 LARGER THAN REQUIRED). If not enough
memory is available for the arrays, a warning message is printed and
$-2$ is returned. The data for {\em xadj} and {\em adjncy} are
determined from the Vertices of the Graph by iterating over each
Vertex from $0$ through {\em numVertex} $-1$. If default options are
specified {\em options[0]} is set to $0$, otherwise $1$ with {\em
options[1:4] = coarsenTo, mType, ipType, rType}. if {\em pType} equals
$1$ {\em PMETIS} is called, otherwise {\em KMETIS} is called. Both are
called with the following arguments: {\em numVertex, xadj,adjncy, 0,
0, \&weightFlag, options, numPart, \&numbering, \&edgecut, partition} 
The colors of the partitions are then set equal to the color indicated
in {\em partition}.  The integer arrays are destroyed and $0$
returned. \\

{\em bool setDefaultOptions(void);}

Sets the default options. \\

\indent {\em bool setOptions(int pType, int mType, int coarsenTo, int 
rType, int ipType);}  \\ 
Sets the options for the partitioning to those passed as
arguments. Then invokes {\em checkOptions()} to see if the options are
valid. HOW ABOUT REFERRINGR TO MANUAL TO SEE WHAT OPTIONS MEAN. \\

\noindent {\bf Private Method}  \\
\indent {\em bool checkOptions(void) const;}
If options are not valid sets the default options. EXPAND ON VALID
OPTIONS OR REFER TO METIS MANUAL. \\
