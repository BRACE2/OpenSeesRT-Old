%File: ~/OOP/element/zeroLength/ZeroLength.tex
%What: "@(#) ZeroLength.tex, revA"

\noindent {\bf Files}   \\
\#include $<\tilde{ }$/element/ZeroLength.h$>$  


\noindent {\bf Class Declaration}  \\
class ZeroLength: public Element 


\noindent {\bf Class Hierarchy} \\
TaggedObject 

MovableObject 

\indent\indent DomainComponent \\
\indent\indent\indent Element \\
\indent\indent\indent\indent {\bf ZeroLength} \\

\noindent {\bf Description}  \\
\indent 
The ZeroLength class represents an element defined by two nodes at the same geometric
location, hence it has zero length.
The nodes are connected by 
of uniaxial materials to represent the force-deformation relationship for the element. 

ZeroLength elements are constructed with a {\em tag} in a domain of {\em dimension} 1, 2, or 3,
connected by nodes {\em Nd1} and {\em Nd2}. 
The vector {\em x} defines the local x-axis for the element and the vector {\em yprime}
lies in the local x-y plane for the element.  The local z-axis is the cross product between 
{\em x} and {\em yprime}, and the local y-axis is the cross product between the local z-axis
and {\em x}.
\\

\noindent {\bf Class Interface} \\
// Constructors 

\indent {\em    ZeroLength(int tag, 			      
	       int dimension,
	       int Nd1, int Nd2, 
	       const Vector\& x,
	       const Vector\& yprime,
	       UniaxialMaterial\& theMaterial,
	       int direction );} \\

\indent {\em    ZeroLength(int tag, 			      
	       int dimension,
	       int Nd1, int Nd2, 
	       const Vector\& x,
	       const Vector\& yprime,
	       int n1dMat,
	       UniaxialMaterial** theMaterial,  
	       const ID\& direction );} \\

{\em    ZeroLength();} \\ 

// Destructor 

{\em    ~ZeroLength();} \\ 


   // public methods to obtain information about dof \& connectivity 

{\em    int getNumExternalNodes(void) const;} 

{\em    const ID \&getExternalNodes(void);} 

\indent {\em    int getNumDOF(void);} \\	
{\em    void setDomain(Domain *theDomain);} 


   // public methods to set the state of the element    

{\em    int commitState(void);} 

\indent {\em    int revertToLastCommit(void);} \\        
\indent {\em    int revertToStart(void);} \\        

   // public methods to obtain stiffness, mass, damping and residual information    

{\em    const Matrix \&getTangentStiff(void);} 

\indent {\em    const Matrix \&getSecantStiff(void);} \\    
\indent {\em    const Matrix \&getDamp(void);} \\    
\indent {\em    const Matrix \&getMass(void);} \\    

\indent {\em    void zeroLoad(void);} \\	
{\em    int addLoad(const Vector \&addP);} 

\indent {\em    int addInertiaLoadToUnbalance(const Vector \&accel);} \\    
{\em    const Vector \&getResistingForce(void);} 

\indent {\em    const Vector \&getResistingForceIncInertia(void);} \\            

   // public methods for element output 

{\em    int sendSelf(int commitTag, Channel \&theChannel);} 

{\em    int recvSelf(int commitTag, Channel \&theChannel, FEM\_ObjectBroker \&theBroker);} 

\indent {\em    int displaySelf(Renderer \&theViewer, int displayMode, float fact);} \\    
\indent {\em    void Print(OPS_Stream \&s, int flag =0);} \\    

{\em    int setResponse(char **argv, int argc, Information \&eleInformation);} 

{\em    int getResponse(int responseID, Information \&eleInformation);} 

    


\noindent {\bf Constructors}  \\
\indent {\em    ZeroLength(int tag, 			      
	       int dimension,
	       int Nd1, int Nd2, 
	       const Vector\& x,
	       const Vector\& yprime,
	       UniaxialMaterial\& theMaterial,
	       int direction );} \\
Construct a ZeroLength element with {\em tag} .
The force-deformation relationship for the element is given by a pointer {\em theMaterial} to a
{\bf UniaxialMaterial} model acting in local {\em direction}.
The local {\em direction} is 1, 2, 3, for translation in the local x, y, z axes or 4, 5, 6 
for rotation about the local x, y, z axes. 
\\

\indent {\em    ZeroLength(int tag, 			      
	       int dimension,
	       int Nd1, int Nd2, 
	       const Vector\& x,
	       const Vector\& yprime,
	       int n1dMat,
	       UniaxialMaterial** theMaterial,  
	       const ID\& direction );} \\
Construct a ZeroLength element with {\em tag} .
The force-deformation relationship is given by the {\em n1dMat} pointers
{\em theMaterial} to {\bf UniaxialMaterial} models. 
{\em direction} is an {\bf ID}  of length {\em n1dMat} that gives
the local direction for each corresponding entry in the array {\em theMaterial}
The local {\em direction} is 1, 2, 3, for translation in the local x, y, z axes or 4, 5, 6 
for rotation about the local x, y, z axes. 
\\

\indent {\em    ZeroLength();} \\ 
This is the constructor invoked by an {\bf FEM\_ObjectBroker} object. It
constructs an empty ZeroLength element with two nodes.
The {\em recvSelf()} method is
invoked on the object for it to set the internal data. 
\\

\noindent {\bf Destructor} \\
\indent {\em    ~ZeroLength();} \\ 
Element destructor deletes memory for storing material model pointers. 
\\

\noindent {\bf Public Methods }  \\
{\em    int getNumExternalNodes(void) const;} 

Returns 2.
\\

{\em    const ID \&getExternalNodes(void);} 

Return {\bf ID} of size $2$ with the node tags defining the element.
\\

\indent {\em    int getNumDOF(void);} \\	
Return the number of DOF for the element, which depends on the dimension of the problem
and the number of DOF associated with each node.
\\

{\em    void setDomain(Domain *theDomain);} 

Initialize element and define data structures.
\\

{\em    int commitState(void);} 

Commit state of element by committing state of materials.
Return 0 if successful, !0 otherwise.
\\

\indent {\em    int revertToLastCommit(void);} \\        
Revert state of element to last commit by reverting to last committed state of materials.
Return 0 if successful, !0 otherwise.
\\

\indent {\em    int revertToStart(void);} \\        
Revert state of element to initial state by reverting to initial state of materials.
Return 0 if successful, !0 otherwise.
\\

{\em    const Matrix \&getTangentStiff(void);} 

Return tangent stiffness matrix for element.
\\

\indent {\em    const Matrix \&getSecantStiff(void);} \\    
Return secant stiffness matrix of element.  The secant stiffness is
defined by the secant for the materials.
\\

\indent {\em    const Matrix \&getDamp(void);} \\    
Return a zero damping matrix.
\\

\indent {\em    const Matrix \&getMass(void);} \\    
Return a zero mass matrix.
\\

\indent {\em    void zeroLoad(void);} \\	
The element has no loads, so this operation has no effect.
\\

{\em    int addLoad(const Vector \&addP);} 

The element has no loads, so this operation has no effect and returns 0.
\\

\indent {\em    int addInertiaLoadToUnbalance(const Vector \&accel);} \\    
The element has no mass, so this operation has no effect and returns 0.
\\

{\em    const Vector \&getResistingForce(void);} 

Return resisting force vector.
\\

\indent {\em    const Vector \&getResistingForceIncInertia(void);} \\            
Return resisting force vector with inertia included.
\\

{\em    int sendSelf(int commitTag, Channel \&theChannel);} 

Send information about element over a channel.
\\

{\em    int recvSelf(int commitTag, Channel \&theChannel, FEM\_ObjectBroker \&theBroker);} 

Receive information about element over a channel.
\\

\indent {\em    int displaySelf(Renderer \&theViewer, int displayMode, float fact);} \\    
Display element.
\\

\indent {\em    void Print(OPS_Stream \&s, int flag =0);} \\    
Print information about element.
\\

{\em    int setResponse(char **argv, int argc, Information \&eleInformation);} 

Set response quantities as "force", "deformation", "material", or "stiff".  Return response
ID  or -1 if error.  Currently, only the one uniaxial material can be set.
\\

{\em    int getResponse(int responseID, Information \&eleInformation);} 

Get response information for {\em responseID}.  Return 0 if successful, -1 otherwise.
\\






