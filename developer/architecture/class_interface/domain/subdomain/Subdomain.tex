% File: ~/domain/domain/subdomain/Subdomain.tex 
% What: "@(#) Subdomain.tex, revA"

UNDER CONSTRUCTION

\noindent {\bf Files}   \\
\#include $<$/domain/subdomain/Subdomain.h$>$  


\noindent {\bf Class Declaration}  \\
class Subdomain: public Domain, public Element  


\noindent {\bf Class Hierarchy} \\
\indent  Domain  \\ 
MovableObject 

TaggedObject 

\indent\indent DomainComponent \\
\indent\indent\indent  Element \\
\indent\indent\indent\indent {\bf Subdomain} \\

\noindent {\bf Description}  \\
\indent  A Subdomain is a Domain that can be an Element in an
enclosing Domain. For this reason, it inherits from both Domain and
Subdomain. It also extends the Domain interface to deal with the
distintion between whether Nodes in the Subdomain are internal to the
Subdomain, or external. An external Node is a Node shared by two
Elements residing in different Subdomains. \\


\noindent {\bf Class Interface}  \\
{\bf Constructors}  

{\em Subdomain(int tag);}  

{\em Subdomain(int tag, 

\indent\indent\indent\indent TaggedObjectStorage \&theInternalNodeStorage,\\
\indent\indent\indent\indent TaggedObjectStorage \&theExternalNodeStorage,\\
\indent\indent\indent\indent TaggedObjectStorage \&theElementsStorage, \\
\indent\indent\indent\indent TaggedObjectStorage \&theLCsStorage, \\
\indent\indent\indent\indent TaggedObjectStorage \&theMPsStorage, \\
\indent\indent\indent\indent TaggedObjectStorage \&theSPsStorage);} \\ \\
// Destructor  

{\em virtual~ $\tilde{}$Subdomain();}  \\ 

// Public Methods modified from Domain class}  

{\em virtual bool addNode(Node *theNodePtr);} 

{\em virtual Node *removeNode(int tag); } 

{\em virtual Node *getNode(int tag); } 

{\em virtual int getNumNodes(void); } 

\indent{\em virtual NodeIter \&getNodes();} \\
{\em virtual void setAnalysis(DomainDecompositionAnalysis *theAnalysis);}

{\em void commit(void);} \\ 

// Public Methods inherited from Element now implemented 

{\em virtual int getNumExternalNodes(void) const;    }

{\em virtual const ID \&getExternalNodes(void);}

{\em virtual int getNumDOF(void);}

\indent {\em virtual void commitState(void);}\\ 
{\em virtual const Matrix \&getStiff(void);}

{\em virtual const Matrix \&getMass(void);    }

{\em virtual const Matrix \&getDamp(void);    }

{\em virtual const Vector \&getResistingForce(void);}

{\em virtual const Vector \&getResistingForceWithInertia(void);}

{\em virtual bool isSubdomain(void);}

\indent {\em int sendSelf(Channel \&theChannel, FEM\_ObjectBroker
\&theBroker);}\\ 
\indent {\em int recvSelf(Channel \&theChannel, FEM\_ObjectBroker
\&theBroker);}\\ \\
 // Public Methods introduced for the Subdomain class 

{\em virtual bool addExternalNode(Node *theNodePtr);} 

\indent{\em virtual NodeIter \&getInternalNodeIter();} \\
\indent{\em virtual const ID \&getExternalNodeIter();} \\
{\em virtual int computeTang(void);}

{\em virtual int computeResidual(void);}

{\em virtual const Matrix \&getTang(void);    }

{\em void setFE\_ElementPtr(FE\_Element *theFEelePtr);}

{\em const Vector \&getLastExternalSysResponse(void);}

{\em virtual int computeNodalResponse(void);}

{\em double getCost(void); }\\ 

// Protected Methods  

\indent{\em FE\_Element *getFE\_ElementPtr(void); }\\



\noindent {\bf Constructors}  \\
{\em Subdomain(int tag);}  

Constructs an empty Subdomain with a number given by {\em tag}. The
storage of the Elements, constraints and loads are handled by the base
Domain class, the storage of the internal and external nodes are
handle by the Subdomain class. For storage of the Nodes objects of
type ArrayOfTaggedObjects are used. The constructors for these objects
take initial sizes of the arrays, the sizes used are:
InternalNodeArray = 8024, ExternalNodeArray = 256. The initial sizes
are not upperbounds on the number of components that can be added. \\


{\em Subdomain(int tag, 

\indent\indent\indent\indent TaggedObjectStorage \&theInternalNodeStorage,\\
\indent\indent\indent\indent TaggedObjectStorage \&theExternalNodeStorage,\\
\indent\indent\indent\indent TaggedObjectStorage \&theElementsStorage, \\
\indent\indent\indent\indent TaggedObjectStorage \&theLCsStorage, \\
\indent\indent\indent\indent TaggedObjectStorage \&theMPsStorage, \\
\indent\indent\indent\indent TaggedObjectStorage \&theSPsStorage);} \\
Constructs an empty Subdomain with a number given by {\em tag}. The
components of the subdomain are stored in the storage objects given in
the argument for the constructor. \\ 

\noindent {\bf Destructor}  \\
{\em virtual~ $\tilde{}$Subdomain();}  

Invokes the destructor on the storage objects for both the internal
and external nodes. \\ 


\noindent {\bf Public Methods modified from Domain class}  \\
{\em virtual bool addNode(Node *theNodePtr);} 

A Method to add the node pointed to by {\em theNoedPtr} to the
Subdomain. This node is an internal node to the domain and is
returned in calls to {\em getNodes()} and {\em getInternalNodes()}. 
Invokes {\em setDomain(this)} on the Node and {\em domainChanged()} on
itself.\\ 

{\em virtual Node *removeNode(int tag); } \\
To remove a Node whose tag is given by {\em tag} from the
Subdomain. Returns $0$ if the Node is not in the Subdoamin, otherwise
it removes the pointer to the node from the storage object, invokes
{\em domainChange()} on itself, and returns the pointer to the
Node. If the Node is external, the destructor is called on the
DummyNode. Invokes {\em domainChanged()} on itself. \\ 

{\em virtual Node *getNode(int tag); } \\
To return a pointer to the Node whose tag is given by {\em tag} from
the Subdomain. Returns $0$ if the Node is not in the Subdoamin,
otherwise returns a pointer to the Node, if external a pointer to the
DummyNode is returned. \\ 

{\em virtual int getNumNodes(void); } \\
Returns the number of external and internal Nodes. \\

\indent{\em virtual NodeIter \&getNodes();} \\
Returns an iter to all nodes that have been added to the subdomain. \\

{\em virtual void setAnalysis(DomainDecompositionAnalysis *theAnalysis);}\\
Sets the corresponding DomainDecompositionAnalysis object to be {\em
theAnalysis}. \\

{\em void commit(void);} \\
invokes the base Domain classes {\em commit()} method. It then goes through
all the Nodes in the Subdomain, invoking {\em commitState()} on the Nodes. \\


\noindent {\bf Public Methods inherited from Element now implemented} \\
{\em virtual int getNumExternalNodes(void) const;    }

Returns the number of external nodes that have been successfully added
to the subdomain as external nodes and have yet to be removed from the
subdomain. \\ 

{\em virtual const ID \&getExternalNodes(void);}\\
Returns an ID containing the tags of all nodes added to the subdomain
as external nodes and have yet to be removed from the subdomain. \\

{\em virtual int getNumDOF(void);}\\
Returns the num of external dof associated with the subdomain, the number
returned is the result of invoking {\em getNumExternalEqn()} on
the DomainDecompAnalysis object associated with the Subdomain. If
no Analysis yet associated with the Subdomain $0$ is returned. \\

{\em virtual void commitState(void);}\\ 
Invokes {\em commit()} on itself. \\
    
{\em virtual const Matrix \&getStiff(void);}\\
For this class does nothing but print an error message. Subtypes may
provide a condensed stiffness matrix, $T^tKT$ corresponding to
external nodes. Returns a zero matrix of dimensions (1x1). \\

{\em virtual const Matrix \&getMass(void);    }\\
For this class does nothing but print an error message. Subtypes may
provide a condensed mass matrix, $T^tMT$ or a mass matrix with zero
diag elements. Returns a zero matrix of dimensions (1x1). \\

{\em virtual const Matrix \&getDamp(void);    }\\
For this class does nothing but print an error message. Subtypes may
provide a condensed damping matrix, $T^tDT$ or a damping matrix
corresponding to some combination of the condensed stifffness and mass
matrices. Returns a zero matrix of dimensions (1x1). \\

{\em virtual const Vector \&getResistingForce(void);}\\
Returns the Vector obtained from invoking {\em getCondensedRHS()} on
the DomainDecompositionAnalysis object. \\

{\em virtual bool isSubdomain(void);}\\
Returns {\em true}. Subtypes can change this. The result of changing
this will be that the corresponding FE\_Elements will ask for the
stiffness, mass and damping matrices to form the tangent and will ask
for the residual to form the residual for the system of equations. \\

{\em int sendSelf(Channel \&theChannel, FEM\_ObjectBroker \&theBroker);}\\
Sends an ID of size 2 to the channel, sending {\em theAnalysis-$>$getClassTag()} and the integer $0$,
if a link exists to a DomainDecompAnalyssi object. 
{\em sendSelf()} is then invoked on theDomainDecompAnalysis object if the link exists. \\

{\em int recvSelf(Channel \&theChannel, FEM\_ObjectBroker \&theBroker);}\\
Receives an ID of size 2 from the channel. If the integer $0$ is sent {\em theBroker}
is asked to construct a new DomainDecompositionAnalysis object using the class tag also sent.
Returns resultt of invoking {\em recvSelf()} on this object. \\

\noindent {\bf Public Methods introduced for the Subdomain class} \\
{\em virtual bool addExternalNode(Node *theNodePtr);} 

A Method to add the node pointed to by {\em theNoedPtr} to the
Subdomain. This node is an external node to the subdomain and is
returned in calls to {\em getExternalNodes()} and {\em
getNodes()}. A DummyNode is created and added to the external nodes storage object. 
Invokes {\em setDomain(this)} on the DummyNode and {\em domainChanged()} on itself.\\

\indent{\em virtual NodeIter \&getInternalNodeIter();} \\
Returns an iter to the internal nodes of the subdomain, nodes that are
added using the {\em addNode()} command.\\

\indent{\em virtual const ID \&getExternalNodeIter();} \\
Returns an ID identifyng the node numbers of all nodes that have been
added using the {\em addExternalNode(Node *)} method. \\


{\em virtual int computeTang(void);}\\
The method first starts a Timer object running. {\em formTang()}, 
is then invoked on the DomainDecompositionAnalysis object. The
Timer is then stopped and the real time is added to the {\em realCost}. 
Returns the result of invoking {\em formTang}. \\

{\em virtual int computeResidual(void);}\\
The method first starts a Timer object running. {\em formResidual()}, 
is then invoked on the DomainDecompositionAnalysis object. The
Timer is then stopped and the real time is added to the {\em realCost}. 
Returns the result of invoking {\em formResidual}. \\

{\em virtual const Matrix \&getTang(void);    }\\
Returns the Matrix obtained from invoking {\em getTangent()} on
the DomainDecompositionAnalysis object. \\


{\em void setFE\_ElementPtr(FE\_Element *theFEelePtr);}\\
Sets the corresponding {\em FE\_Element} to be that poited to by {\em
theFEelePtr} . \\

{\em const Vector \&getLastExternalSysResponse(void);}\\
Returns the Vector obtained by calling {\em getLastSysResponse()} on
the associated FE\_Element. \\

{\em virtual int computeNodalResponse(void);}\\
To set the nodal responses for the nodes in the subdomain. Returns the
result of invoking {\em computeInternalResponse()} on the DomainDecomposition
analysis object associated with the subdomain.\\ 

{\em double getCost(void); }\\
Returns the current value of {\em realCost}, restting the value of {\em realCost} to be 0.0. The
value of {\em realCost} is added to when {\em computeTang()} and {\em computeREsidual} are invoked on the
Subdomain. \\

\noindent {\bf Protected Member Functions}  \\
\indent{\em FE\_Element *getFE\_ElementPtr(void); }\\
Returns a pointer to the last FE\_Element set using {\em
setFE\_ElementPtr}. If no FE\_Element has been set $0$ is returned. \\






