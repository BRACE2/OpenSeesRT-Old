%File: ~/OOP/material/ElasticPPModel.tex
%What: "@(#) ElasticPPModel.tex, revA"

UNDER CONSTRUCTION. \\

\noindent {\bf Files}   \\
\#include $<\tilde{ }$/material/ElasticPPModel.h$>$  


\noindent {\bf Class Declaration}  \\
class ElasticPPMaterial: public MaterialModel 


\noindent {\bf Class Hierarchy} \\
TaggedObject 

MovableObject 

\indent\indent MaterialModel \\
\indent\indent\indent UniaxialMaterial \\
\indent\indent\indent\indent {\bf ElasticPPMaterial} \\

\noindent {\bf Description}  \\
\indent ElasticPPMaterial provides the abstraction of an elastic
perfectly plastic uniaxial material, i.e. the stress-strain
relationship is given by the linear equation $\sigma = E * \epsilon$
while the material is elastic and $| \sigma = E * \epsilon_p |$  while
the material is undergoing plastic deformation. SOME MORE THEORY. \\

\noindent {\bf Class Interface} \\
// Constructor 

{\em ElasticPPMaterial(int tag, double E, double ep);}  \\ 

// Destructor 

{\em $\tilde{ }$ElasticPPMaterial();}\\ 

// Public Methods 

{\em int setTrialStrain(double strain); } 

{\em double getStress(void); } 

{\em double getTangent(void); } 

{\em int commitState(void); } 

{\em int revertToLastCommit(void); } 

{\em int revertToStart(void); } 

{\em UniaxialMaterial *getCopy(void); } \\ 

// Public Methods for Output

{\em    int sendSelf(int commitTag, Channel \&theChannel); }

\indent {\em    int recvSelf(int commitTag, Channel \&theChannel, 
		 FEM\_ObjectBroker \&theBroker); }\\
{\em    void Print(OPS_Stream \&s, int flag =0);} 


