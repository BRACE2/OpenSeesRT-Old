% File: ~/domain/loadcase/Load.tex 

\noindent {\bf Files}   \\
\#include $<\tilde{ }$/domain/load/Load.h$>$  


\noindent {\bf Class Declaration}  \\
class Load: public DomainComponent  


\noindent {\bf Class Hierarchy} \\
TaggedObject 

MovableObject 

\indent\indent DomainComponent \\
\indent\indent\indent {\bf Load} \\

\noindent {\bf Description}  \\
\indent Load is an abstract base class. A Load object is used to add
load to the domain. The Load class defines one method in its interface
{\em applyLoad()}, a method all subclasses must implement. \\


\noindent {\bf Class Interface}  \\
// Constructor  

{\em Load(tag, int classTag);}  \\ 

// Destructor  

{\em virtual $\tilde{ }$ Load();} \\ 

// Public Methods   

{\em virtual void applyLoad(loadFactor) = 0;} 

{\em virtual void setLoadPatternTag(int loadPaternTag);}

{\em virtual int  getLoadPatternTag(void) const;}


\noindent {\bf Constructor}  \\
{\em Load(tag, int classTag);}  

Constructs a load with a tag given by {\em tag} and a class tag is
given by {\em classTag}. These are passed to the DomainComponent constructor. \\

\noindent {\bf Destructor}  \\
{\em virtual~$\tilde{}$ Load();} 


\noindent {\bf Public Methods }  \\
{\em virtual void applyLoad(double loadFactor) = 0;} 

The load object is to add {\em loadFactor} times the load to the
corresponding residual value at its associated element(s) or node(s). \\

{\em virtual void setLoadPatternTag(int loadPaternTag);}

To set the tag of the enclosing load pattern for the load to be 
{\em loadPatternTag}. \\

{\em virtual int  getLoadPatternTag(void) const;}

To return the current load pattern tag associated with the load. If no
load pattern tag has been set $-1$ is returned.

