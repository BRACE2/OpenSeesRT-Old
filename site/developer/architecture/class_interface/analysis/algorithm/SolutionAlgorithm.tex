%File ~/OOP/analysis/algorithm/SolutionAlgorithm.tex
%What: "@(#) SolutionAlgorithm.tex, revA"

\noindent {\bf Files}   \\
\#include $<\tilde{ }$/analysis/algorithm/SolutionAlgorithm.h$>$  


\noindent {\bf Class Declaration}  \\
class SolutionAlgorithm: public MovableObject  


\noindent {\bf Class Hierarchy} \\
MovableObject 

\indent\indent {\bf SolutionAlgorithm} \\

\noindent {\bf Description} \\
\indent The SolutionAlgorithm class is an abstract base class. Its purpose
is to define the interface common among all its subclasses. A
SolutionAlgorithm object performs the steps in the analysis by specifying
the sequence of operations to be performed by members in the analysis
aggregation.\\


\noindent {\bf Class Interface} \\
// Constructor 

{\em SolutionAlgorithm(int classTag);}\\  

// Destructor 

{\em virtual~ $\tilde{}$SolutionAlgorithm();}\\  

// Public Methods  

{\em virtual int domainChanged(void); } 

{\em virtual  int  addRecorder(Recorder \&theRecorder);}

{\em virtual int record(int track); } 

{\em virtual int playback(int track); } 



\noindent {\bf Constructor} \\
{\em SolutionAlgorithm(int classTag);}

The integer {\em classTag} is passed to the MovableObject classes
constructor. \\

\noindent {\bf Destructor} \\
{\em virtual~ $\tilde{}$SolutionAlgorithm();}

Invokes the destructor on any recorder object added to the
SolutionAlgorithm and releases memory used to hold pointers to the
recorder objects. \\


\noindent {\bf Public Methods}  \\
{\em virtual int domainChanged(void); } 

Is called by the Analysis if the domain changes. It is called after
{\em domainChange()} has been called on the ConstraintHandler,
DOF\_Numberer and the Integrator and after {\em setSize()} has been
called on the SystemOfEqn object. For base class nothing is done and
$0$ is returned. The subclasses can provide their own implementation
of this method if anything needs to be done, e.g. memory allocation,
To return $0$ if successful, a negative number if not. \\


{\em  virtual int  addRecorder(Recorder \&theRecorder);}

To add a recorder object {\em theRecorder} to the
SolutionAlgorithm. returns $0$ if successful, a warning message and a
$-1$ is returned if not enough memory is available. \\

{\em virtual int record(int track); } 

To invoke {\em record(track)} on any Recorder objects which have been added to the
SolutionAlgorithm. \\

{\em virtual int playback(int track); } 

To invoke {\em playback(track)} on any Recorder objects which have been added to the
SolutionAlgorithm. \\
