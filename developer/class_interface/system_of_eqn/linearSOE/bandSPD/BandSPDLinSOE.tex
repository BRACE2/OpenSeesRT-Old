%File: ~/OOP/system_of_eqn/linearSOE/bandSPD/BandSPDLinSOE.tex
%What: "@(#) BandSPDLinSOE.tex, revA"

\noindent {\bf Files}   \\
\#include $<\tilde{ }$/system\_of\_eqn/linearSOE/bandSPD/BandSPDLinSOE.h$>$  


\noindent {\bf Class Declaration}  \\
class BandSPDLinSOE: public LinearSOE 


\noindent {\bf Class Hierarchy} \\
MovableObject 

\indent\indent SystemOfEqn \\
\indent\indent\indent LinearSOE \\
\indent\indent\indent\indent {\bf BandSPDLinSOE} \\

\noindent {\bf Description}  \\
\indent BandSPDLinSOE is class which is used to store a banded symmetric system
with ku superdiagonals. The $A$ matrix is stored in a
1d double array with (ku+1)*n elements, where n is the size of the
system. $A_{i,j}$ is stored at location $(ku+1+i-j) +
j*(ku+1)$, where $i$ and $j$ range from $0$ to$n-1$, i.e. C
notation. For example when $n=5$, $ku = 2$: 


$$

$$

\left[
\begin{array}{ccccc}
a_{0,0} & a_{0,1}  & a_{0,1} & 0 & 0 \\
a_{1,0} & a_{1,1} & a_{1,2} & a_{1,3} & 0 \\
a_{2,0} & a_{2,1} & a_{2,2} & a_{2,3} & a_{2,4}  \\
0 & a_{3,1} & a_{3,2} & a_{3,3} & a_{3,4} \\
0 & 0 & a_{4,2} & a_{4,3} & a_{4,4} \\
\end{array}
\right] 

$$

$$


is stored in:


$$

$$

\left[
\begin{array}{cccccccccccccccccccc}
* & * & a_{0,0} & * & a_{0,1}  & a_{1,1} & a_{0,2} & a_{1,2} & a_{2,2} &
a_{1,3} & a_{2,3} & a_{3,3} & a_{2,4} & a_{3,4} & a_{4,4}\\
\end{array}
\right] 

$$

$$


The $X$ and $B$ vectors are stored in 1d double arrays of length $N$. \\

\noindent {\bf Interface}  \\
\indent\indent {// Constructors} \\
\indent\indent {\em BandSPDLinSOE(BandSPDLinSolver \&theSolver);}  \\
\indent\indent {\em BandSPDLinSOE(int N, int ku, BandSPDLinSolver
\&theSolver);}\\ \\
\indent\indent {// Destructor} \\
\indent\indent {\em  $\tilde{}$BandSPDLinSOE();}\\ \\
\indent\indent {// Public Methods }  \\
\indent\indent {\em  int setBandSPDSolver(BandSPDLinSolver \&newSolver);}\\
\indent\indent {\em int setSize(const Graph \&theGraph) =0; } \\
\indent\indent {\em int getNumEqn(void) =0; } \\
\indent\indent {\em int addA(const Matrix \&theMatrix, const ID \& loc,
doublefact = 1.0) =0;} \\
\indent\indent {\em int addB(const Vector \& theVector, const ID \& loc,
double fact = 1.0) =0;} \\
\indent\indent {\em int setB(const Vector \& theVector, 
double fact = 1.0) =0;} \\
\indent\indent {\em void zeroA(void) =0;} \\
\indent\indent {\em void zeroB(void) =0;} \\
\indent\indent {\em const Vector \&getX(void) = 0;} \\
\indent\indent {\em const Vector \&getB(void) = 0;} \\
\indent\indent {\em double normRHS(void) =0;} \\
\indent\indent {\em void setX(int loc, double value) =0;}\\
\indent\indent {\em int sendSelf(int commitTag, Channel \&theChannel);}\\ 
\indent\indent {\em int recvSelf(int commitTag, Channel \&theChannel,
FEM\_ObjectBroker \&theBroker);}\\ 


\noindent {\bf Constructors}  \\
{\em BandSPDLinSOE(BandSPDLinSolver \&solver);}  

The {\em solver} and a unique class tag (defined in $<$classTags.h$>$)
are passed to the LinearSOE constructor. The system size is set
to $0$ and the matrix $A$ is marked as not having been factored. Invokes
{\em setLinearSOE(*this)} on the {\em solver}. No memory is
allocated for the 3 1d arrays. \\  

{\em BandSPDLinSOE(int N, int ku, BandSPDLinSolver \&theSolver);}\\
The {\em solver} and a unique class tag (defined in $<$classTags.h$>$)
are passed to the LinearSOE constructor. 
Sets the size of the system to $N$, the number of superdiagonals to 
{\em ku}. Allocates memory for the arrays; if not enough
memory is available a warning message is printed and the system size
is set to $0$. Sets the solver to be called when solving the
equations to {\em theSolver}. Invokes {\em setLinearSOE(*this)} and
{\em setSize()} on the {\em theSolver}, printing a warning message if
{\em setSize()} returns a negative number. Also creates Vector objects
for $x$ and $b$ using the {\em (double *,int)} Vector constructor. \\


\noindent {\bf Destructor} \\
\indent {\em  $\tilde{}$BandSPDLinSOE();}\\ 
Calls delete on any arrays created. \\

\noindent {\bf Public Methods }  \\
{\em  int setBandSPDSolver(BandSPDLinSolver \&newSolver);}

Invokes {\em setLinearSOE(*this)} on {\em newSolver}.
If the system size is not equal to $0$, it also invokes {\em setSize()}
on {\em newSolver}, printing a warning and returning the returned value if this
method returns a number less than $0$. Finally it returns the result
of invoking the LinearSOE classes {\em setSolver()} method. \\

{\em int getNumEqn(void) =0; } 

A method which returns the current size of the system. \\

\indent {\em  int setSize(const Graph \&G); } \\ 
The size of the system is determined by looking at the adjacency ID of
each Vertex in the Graph object {\em G}. This is done by first setting
{\em ku} equal to $0$ and then checking for each Vertex
in {\em G}, whether any of the vertex tags in the Vertices adjacency
ID results in {\em ku} being increased. Knowing {\em ku} and the size
of the system (the number of Vertices in {\em G}, a check to see if
the previously allocated 1d arrays for $A$, $x$ and $b$ are large
enough. If the memory portions allocated for the 1d arrays are not big
enough, the old space is returned to the heap and new space is
allocated from the heap. Prints a warning message if not enough
memory is available on the heap for the 1d arrays and returns a
$-1$. If memory is available, the components of the arrays are zeroed
and $A$ is marked as being unfactored. If the system size has
increased, new Vector objects for $x$ and $b$ using the {\em (double
*,int)} Vector constructor are created. Finally, the result of
invoking {\em setSize()} on the associated Solver object is
returned. \\ 


\indent {\em int addA(const Matrix \&M, const ID \& loc,
doublefact = 1.0) =0;} \\
First tests that {\em loc} and {\em M} are of compatible sizes; if not
a warning message is printed and a $-1$ is returned. The LinearSOE
object then assembles {\em fact} times the Matrix {\em 
M} into the matrix $A$. The Matrix is assembled into $A$ at the
locations given by the ID object {\em loc}, i.e. $a_{loc(i),loc(j)} +=
fact * M(i,j)$. If the location specified is outside the range,
i.e. $(-1,-1)$ the corresponding entry in {\em M} is not added to
$A$. If {\em fact} is equal to $0.0$ or $1.0$, more efficient steps
are performed. Returns $0$.  \\


{\em int addB(const Vector \& V, const ID \& loc,
double fact = 1.0) =0;} \\
First tests that {\em loc} and {\em V} are of compatible sizes; if not
a warning message is printed and a $-1$ is returned. The LinearSOE
object then assembles {\em fact} times the Vector {\em V} into
the vector $b$. The Vector is assembled into $b$ at the locations
given by the ID object {\em loc}, i.e. $b_{loc(i)} += fact * V(i)$. If a
location specified is outside the range, e.g. $-1$, the corresponding
entry in {\em V} is not added to $b$. If {\em fact} is equal to $0.0$,
$1.0$ or $-1.0$, more efficient steps are performed. Returns $0$. \\


{\em int setB(const Vector \& V, double fact = 1.0) =0;} \\
First tests that {\em V} and the size of the system are of compatible
sizes; if not a warning message is printed and a $-1$ is returned. The
LinearSOE object then sets the vector {\em b} to be {\em fact} times
the Vector {\em V}. If {\em fact} is equal to $0.0$, $1.0$ or $-1.0$,
more efficient steps are performed. Returns $0$. \\ 

{\em void zeroA(void) =0;} \\
Zeros the entries in the 1d array for $A$ and marks the system as not
having been factored. \\

{\em void zeroB(void) =0;} \\
Zeros the entries in the 1d array for $b$. \\

{\em const Vector \&getX(void) = 0;} \\
Returns the Vector object created for $x$. \\

{\em const Vector \&getB(void) = 0;} \\
Returns the Vector object created for $b$. \\

{\em double normRHS(void) =0;} \\
Returns the 2-norm of the vector $x$. \\

{\em void setX(int loc, double value) =0;}\\
If {\em loc} is within the range of $x$, sets $x(loc) = value$. \\

\indent {\em int sendSelf(int commitTag, Channel \&theChannel);}\\ 
Returns $0$. The object does not send any data or connectivity
information as this is not needed in the finite element design. \\

\indent {\em int recvSelf(int commitTag, Channel \&theChannel, FEM\_ObjectBroker
\&theBroker);}\\ 
Returns $0$. The object does not receive any data or connectivity
information as this is not needed in the finite element design.
