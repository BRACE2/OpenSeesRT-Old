%File: ~/OOP/element/Node.tex
%What: "@(#) Node.tex, revA"

NEW METHOD POSSIBLY NEEDED TO SPECIFY THE NUMBER OF PREVIOUSLY
COMMITTED RESPONSES TO KEEP .. NEEDED FOR EXPLICIT DYNAMIC INTEGRATORS\\

\noindent {\bf Files}   \\
\#include $<$/domain/node/Node.h$>$  


\noindent {\bf Class Declaration}  \\
class Node: public DomainComponent  


\noindent {\bf Class Hierarchy} \\
TaggedObject 

MovableObject 

\indent\indent DomainComponent \\
\indent\indent\indent {\bf Node} \\

\noindent {\bf Description}  \\
\indent Nodes are points in space connected by the elements. Nodes
have original position, trial displacement, velocity and acceleration, 
and committed displacement, velocity and acceleration (the last
committed trial quantities). Nodes also store information about any
load acting on the node, nodal mass and the nodal participation
matrix. In addition, each Node object keeps track of it's associated
DOF\_Group object. The Node interface provides methods to set and
retrieve these quantities.\\  


\noindent {\bf Class Interface}  \\
\indent\indent // Constructors  \\
\indent\indent {\em Node(int classTag);}  \\
\indent\indent {\em Node(int tag, int classTag);}  \\
\indent\indent {\em Node(int tag, int ndof, double Crd1);}  \\
\indent\indent {\em Node(int tag, int ndof, double Crd1, double Crd2);}  \\
\indent\indent {\em Node(int tag, int ndof, double Crd1, double Crd2, double Crd3);}\\
\indent\indent {\em Node(const Node \&theCopy);}\\ \\
\indent\indent // Destructor  \\
\indent\indent {\em virtual $\tilde{ }$ Node();} \\ \\
\indent\indent // Public Methods dealing with DOF at the Node\\
\indent\indent {\em virtual int  getNumberDOF(void) const;}\\
\indent\indent {\em virtual void setDOF\_GroupPtr(DOF\_Group *theDOF\_Grp);} \\
\indent\indent {\em virtual DOF\_Group *getDOF\_GroupPtr(void);} \\ \\
\indent\indent // Public Method for obtaining nodal coordinates\\
\indent\indent {\em virtual const Vector \&getCrds(void) const;}  \\ \\
\indent\indent // Public Method for obtaining committed and trial responses\\ 
\indent\indent {\em  virtual const Vector \&getDisp(void) ;}  \\
\indent\indent {\em  virtual const Vector \&getVel(void) ;}  \\
\indent\indent {\em  virtual const Vector \&getAccel(void) ;}  \\
\indent\indent {\em  virtual const Vector \&getTrialDisp(void) ;}  \\
\indent\indent {\em  virtual const Vector \&getTrialVel(void) ;}  \\
\indent\indent {\em  virtual const Vector \&getTrialAccel(void) ;}  \\ 
\indent\indent {\em  virtual const Vector \&getIncrDisp(void) ;}  \\\\
\indent\indent // Public Method for updating trial responses\\
\indent\indent {\em virtual int setTrialDisp(const Vector \&newTrialDisp);} \\
\indent\indent {\em virtual int setTrialVel(const Vector \&newTrialVel);} \\
\indent\indent {\em virtual int setTrialAccel(const Vector \&newTrialAccel);} \\
\indent\indent {\em virtual int incrTrialDisp(const Vector \&trialIncrDisp);} \\
\indent\indent {\em virtual int incrTrialVel(const Vector \&trialIncrVel);} \\
\indent\indent {\em virtual int incrTrialAccel(const Vector \&trialIncrAccel);} \\ \\
\indent\indent // Public Method for setting and obtaining unbalanced load\\
\indent\indent {\em virtual void zeroUnbalancedLoad(void);} \\
\indent\indent {\em virtual int addUnbalancedLoad(const Vector
\&additionalLoad, double fact = 1.0);} \\ 
\indent\indent {\em virtual int addInertiaLoadToUnbalance(const Vector
\&accel, double fact = 1.0);} \\ 
\indent\indent {\em virtual const Vector \&getUnbalancedLoad(void);} \\
\indent\indent {\em virtual const Vector \&getUnbalancedLoadIncInertia(void);} \\ \\
\indent\indent // Public Method for setting state \\
\indent\indent {\em virtual int commitState(void);} \\
\indent\indent {\em virtual int revertToLastCommit(void);} \\
\indent\indent {\em virtual int revertToStart(void);} \\ \\
\indent\indent // Public Method for dynamic and modal analysis\\
\indent\indent {\em virtual const Matrix \&getMass(void) ;}\\
\indent\indent {\em virtual int setMass(const Matrix \&mass);} \\
\indent\indent {\em virtual int setNumColR(int numCol);} \\
\indent\indent {\em virtual int setR(int row, int col, double Value);} \\
\indent\indent {\em virtual const Vector \& getRV(const Vector \&V);}\\ \\
\indent\indent // Public Method for Output \\
\indent\indent {\em virtual int sendSelf(int commitTag, Channel
\&theChannel);} \\
\indent\indent {\em virtual int recvSelf(int commitTag, Channel \&theChannel,
FEM\_ObjectBroker \&theBroker);} \\ 
\indent\indent {\em void Print(OPS_Stream \&s, int flag = 0);} \\
\indent\indent {\em int displaySelf(Renderer \&theRenderer, int flag,
float fact);} \\



\noindent {\bf Constructors}  \\
{\em Node(int classTag);}  

To construct a node which has no data, other than the {\em classTag}
identifier; $0$ and {\em classTag} are passed to the DomainComponent
constructor. This is the constructor called by an
FEM\_ObjectBroker. The data must be filled in subsequently by a call
to {\em recvSelf()}. \\ 

{\em Node(int tag, int classTag);}  

To construct a node whose unique integer among nodes in the
domain is given by {\em tag} and whose classTag is given by {\em
classTag}. This constructor can be used by 
subclasses who wish to handle their own data management. \\ 

{\em Node(int tag, int ndof, double Crd1);}  

To construct a node for 1d problems whose unique integer among nodes in the
domain is given by {\em tag} and whose original position in 1d space
is given by (Crd1). With the node is associated {\em ndof} number
of degrees of freedom. The class tag is NOD\_TAG\_Node (defined in
classTags.h). A Vector object is created to hold the coordinates. No
storage objects are created to hold the trial and committed response
quantities, mass, load quantities; these are created as needed to
reduce the memory demands on the system in certain situations. \\ 

{\em Node(int tag, int ndof, double Crd1, double Crd2);}  

To construct a node for 2d problems whose unique integer among nodes in the
domain is given by {\em tag} and whose original position in 2d space
is given by (Crd1,Crd2). With the node is associated {\em ndof} number
of degrees of freedom. The class tag is NOD\_TAG\_Node. A Vector object
is created to hold the coordinates. No
storage objects are created to hold the trial and committed response
quantities, mass, load quantities; these are created as needed to
reduce the memory demands on the system in certain situations. \\ 

{\em Node(int tag, int ndof, double Crd1, double Crd2, double Crd3);}

To construct a node for 3d problems whose unique integer among nodes in the
domain is given by {\em tag} and whose original position in 3d space
is given by (Crd1,Crd2,Crd3). With the node is associated {\em ndof} number
of degrees of freedom. The class tag is NOD\_TAG\_Node. A Vector object
is created to hold the coordinates. No
storage objects are created to hold the trial and committed response
quantities, mass, load quantities; these are created as needed to
reduce the memory demands on the system in certain situations. \\ 



{\em Node(const Node \&theCopy);}

To construct a node which is an exact copy of {\em theCopy}. \\

\noindent {\bf Destructor}  \\
{\em virtual~$\tilde{}$ Node();} 

Invokes the destructor on all the storage objects created to hold the coordinates,
response quantities, mass and load quantities. \\

\noindent {\bf Public Member Functions }  \\
{\em virtual int  getNumberDOF(void) const;}

Returns the number of degrees-of-freedom, {\em ndof}, associated with
the node. \\

{\em virtual void setDOF\_GroupPtr(DOF\_Group *theDOF\_Grp);} 

Each node, when involved with an analysis, will be associated with a
DOF\_Group object. It is the DOF\_Group that contains the ID of equation
numbers. When invoked this method sets the pointer to that DOF\_Group object. \\

{\em virtual DOF\_Group *getDOF\_GroupPtr(void);} \\
Method which returns a pointer to the DOF\_Group object that was set
using {\em setDOF\_GroupPtr}. If no pointer has been set a $0$ is
returned. \\

{\em virtual const Vector \&getCrds(void) const;}  \\
Returns the original coordinates in a Vector. The size of the vector
is 2 if node object was created for a 2d problem and the size is 3 if
created for a 3d problem. \\

{\em  virtual const Vector \&getDisp(void) ;}  \\
Returns the last committed displacement as a Vector, the vector of
size {\em ndof}. If no Vector has yet been allocated, two Vector
objects are created to store the committed and trial response
quantities created; if not enough space is available an error message
is printed and program terminated. \\ 

{\em  virtual const Vector \&getVel(void) ;}  \\
Returns the last committed velocity as a Vector, the vector of size
{\em ndof}. If no Vector has yet been allocated, two Vector
objects are created to store the committed and trial response
quantities created; if not enough space is available an error message
is printed and program terminated. \\ 


{\em  virtual const Vector \&getAccel(void) ;}  \\
Returns the last committed acceleration as a Vector, the vector of
size {\em ndof}. If no Vector has yet been allocated, two Vector 
objects are created to store the committed and trial response
quantities created; if not enough space is available an error message
is printed and program terminated. \\ 

{\em  virtual const Vector \&getTrialDisp(void) ;}  \\
Returns the current trial displacements as a Vector, the vector of size {\em ndof}.
If no Vector has yet been allocated, a new Vector is created and returned;
if not enough space is available an error message is printed and the
program is terminated. \\ 


{\em  virtual const Vector \&getTrialVel(void) ;}  \\
Returns the current trial velocities as a Vector, the vector of size {\em ndof}.
If no Vector has yet been allocated, a new Vector is created and returned;
if not enough space is available an error message is printed and the
program is terminated. \\ 

{\em  virtual const Vector \&getTrialAccel(void) ;}  \\
Returns the current trial accelerations as a Vector, the vector of size {\em ndof}.
If no Vector has yet been allocated, a new Vector is created and returned;
if not enough space is available an error message is printed and the
program is terminated. \\ 


{\em  virtual const Vector \&getIncrDisp(void) ;}  \\
Returns the incremental displacement as a Vector. The incremental displacement is 
equal to the difference between the current trial displacement and committed
displacement (trial - committed).
If no Vector has yet been allocated, three Vector 
objects are created to store the committed, trial and incremental response
quantities; if not enough space is available an error message
is printed and program terminated. \\ 


{\em virtual int setTrialDisp(const Vector \&newTrialDisp);} \\
Sets the current trial displacement to be that given by
{\em newTrialDisp}. Sets th incremental displacement to be trial $-$ committtd.
If no space has yet been allocated for
the trial displacements, two Vector objects are now created to store
the trial and committed response quantities; if not enough memory 
is available on the heap to create these new Vectors an error message
is printed and the program is terminated. Returns $0$ if successful,
an error message is printed and a $-2$ is returned if {\em
newTrialDisp} is not of size {\em ndof}. \\


{\em virtual int setTrialVel(const Vector \&newTrialVel);} \\
Sets the current trial velocity to be that given by
{\em newTrialVel}. If no space has yet been allocated for
the trial velocities, two Vector objects are now created to store
the trial and committed response quantities; if not enough memory 
is available on the heap to create these new Vectors an error message
is printed and the program is terminated. Returns $0$ if successful,
an error message is printed and a $-2$ is returned if {\em
newTrialVel} is not of size {\em ndof}. \\

{\em virtual int setTrialAccel(const Vector \&newTrialAccel);} \\
Sets the current trial acceleration to be that given by
{\em newTrialAccel}. If no space has yet been allocated for
the trial accelerations, two Vector objects are now created to store
the trial and committed response quantities; if not enough memory 
is available on the heap to create these new Vectors an error message
is printed and the program is terminated. Returns $0$ if successful,
an error message is printed and a $-2$ is returned if {\em
newTrialAccel} is not of size {\em ndof}. \\

{\em virtual int incrTrialDisp(const Vector \&trialIncrDisp);} \\
Sets the current trial displacement to be that given by the addition
of the last trial displacement, assumed $0$ if not yet set, and {\em
trialIncrDisp}. Increments the incremental displacement by {\em trialIncrDisp}.
If no space has yet been allocated for the displacements, three Vector objects 
are now created to store
the trial, committed and incremental response quantities; if not enough memory 
is available on the heap to create these new Vectors an error message
is printed and the program is terminated. Returns $0$ if successful,
an error message is printed and a $-2$ is returned if {\em
trialIncrDisp} is not of size {\em ndof}. \\

{\em virtual int incrTrialVel(const Vector \&trialIncrVel);} \\
Sets the current trial velocity to be that given by the addition
of the last trial velocity, assumed $0$ if not yet set, and {\em
trialIncrVel}. If no space has yet been allocated for
the trial velocities, two Vector objects are now created to store
the trial and committed response quantities; if not enough memory 
is available on the heap to create these new Vectors an error message
is printed and the program is terminated. Returns $0$ if successful,
an error message is printed and a $-2$ is returned if {\em
trialIncrVel} is not of size {\em ndof}. \\

{\em virtual int incrTrialAccel(const Vector \&trialIncrAccel);} \\
Sets the current trial Acceleration to be that given by the addition
of the last trial Acceleration, assumed $0$ if not yet set, and
{\em trialIncrAccel}.  If no space has yet been allocated for
the trial accelerations, two Vector objects are now created to store
the trial and committed response quantities; if not enough memory 
is available on the heap to create these new Vectors an error message
is printed and the program is terminated. Returns $0$ if successful,
an error message is printed and a $-2$ is returned if {\em
trialIncrAccel} is not of size {\em ndof}. \\

{\em virtual void zeroUnbalancedLoad(void);} \\
Causes the node to zero out its unbalanced load vector. \\

{\em virtual int addUnbalancedLoad(const Vector \&additionalLoad, double fact);} \\
The Node is responsible for adding {\em fact} times {\em
additionalLoad} to the current unbalanced load at the Node. If {\em
additionalLoad} is not of size {\em ndof} no load is added, an error
message is printed and a $-1$ is returned. If no space has yet been
allocated for the unbalanced load a new Vector is now created;  
if not enough space is available for this Vector an error message is
printed and the program is terminated. Returns $0$ if successful.  \\ 

\indent {\em virtual int addInertiaLoadToUnbalance(const Vector
\&accel, double fact = 1.0);} \\ 
To add {\bf minus} {\em fact} times the product $M * R * accel$ to the
current unbalanced load. Nothing is done if no mass or R matrix have
been set. Prints a warning and returns a $-1$ if the size of accel and
the number of columns in $R$ are not the same. If no space has yet been
allocated for the unbalanced load a new Vector is now created;  
if not enough space is available for this Vector an error message is
printed and the program is terminated. Returns $0$ if successful.  \\ 

{\em virtual const Vector \&getUnbalancedLoad(void);} \\
Returns the current unbalanced load. If no space has yet been
allocated for the unbalanced load a new Vector of size {\em numDOF} is
now created; if not enough space is available for this Vector an error
message is printed and the program is terminated. \\ 

{\em virtual const Vector \&getUnbalancedLoadIncInertia(void);} \\
Returns the current unbalanced load Vector, as defined above, MINUS the product
of the nodes mass matrix and the trial nodal accelerations. The result
is saved in another vector which is returned. If no space has yet been
allocated for this new Vector, a Vector of size {\em numDOF} is now
created; if not enough space is available for this Vector an error
message is printed and the program is terminated. \\ 

{\em virtual int commitState(void);} \\
Causes the node to set the committed model displacements, velocities and
accelerations to be equal to the current trial displacements, velocities and
accelerations. The incremental displacement is set to $0$. No assignment is done 
for any of the quantities for
which no memory has been allocated. Returns $0$.\\

{\em virtual int revertToLastCommit(void);} \\
Causes the node to set the trial nodal displacements, velocities and
accelerations to be equal to the current committed displacements, velocities and
accelerations. The incremental displacement is set to $0$. No assignment is done 
for any of the trial quantities for which no memory has been allocated. Returns $0$.\\

{\em virtual int revertToStart(void);} \\
Causes the node to set the trial and committed nodal displacements,
velocities and accelerations to zero. No assignment is done for any of
the trial quantities for which no memory has been allocated. Returns $0$.\\


{\em virtual const Matrix \&getMass(void) ;}\\
Returns the mass matrix set for the node, which is a matrix of size
{\em ndof,ndof}. This matrix is equal to that set in {\em setMass()}
or zero if {\em setMass()} has not been called. If no storage space
has been allocated for the mass, a matrix is now created. An error
message is printed and the program terminated if no space is available
on the heap for this matrix.\\

{\em virtual int setMass(const Matrix \&mass);} \\
Sets the value of the mass at the node. A check is made to ensure that
the {\em mass} has the same dimensions of the mass matrix associated with the
Node; if incompatible size an error message is printed and -1 returned. If no
mass matrix yet allocated, one is created; if no space is available an
error message is printed and the program terminated. Returns 0 if successful. \\

{\em virtual int setNumColR(int numCol);} 

Creates a Matrix to store the R matrix. The matrix is of dimension
{\em ndof} and {\em numCol}. If not enough space is available for this
matrix an error message is printed and the program is terminated. Zeros the
matrix R and returns $0$ if successful. \\

{\em virtual int setR(int row, int col, double Value);} 

Sets the {\em \(row,col\)} entry of R to be equal to {\em Value}. If
no matrix R has been specified or the position in R is out of range a
warning message is printed and a $-1$ is returned. Returns $0$ if
successful. \\


{\em virtual const Vector \& getRV(const Vector \&V);} 

This is a method provided for Element objects, the Node object returns
the product of the matrix $R$ and the vector $V$. If the matrix 
and vector are of inappropriate size a warning message is printed and
a zero vector is returned. \\

{\em virtual int sendSelf(int commitTag, Channel \&theChannel);} \\
Causes the Node object to send the data needed to init itself on a
remote machine to the Channel object {\em theChannel}. 
The data sent includes the tag, number of dof, coordinates, committed
response quantities, unbalanced load, mass and participation matrix. 
To do this the Node creates an ID object into which it stores its tag,
the {\em ndof} and a flag indicating whether any additional
information, i.e. mass, response quantities also need to be sent. In
addition four database tags are also included in this ID object. The
database tags, if not already obtained, are requested from the Channel
object (these are needed as each object can only store a single object
of a particular size using it's own database tags -- additional tags
are needed when multiple objects of the same size are needed.
The objects that have been created are then sent. \\

{\em virtual int recvSelf(int commitTag, Channel \&theChannel,
FEM\_ObjectBroker \&theBroker);} \\ 
Invoked on a remote machine to read its data that was sent by a node
object in another actor when {\em sendSelf()} was invoked. As in {\em
sendSelf()}, the Node object creates an ID object. It asks the Channel
object to fill this object with data. Based on the data it creates
Matrix and Vector objects to store the Nodes data and asks the Channel
object to fill these with data. The data placed here by the Channel
object correspond to the data put there by the sending Node object.\\


{\em void Print(OPS_Stream \&s, int flag = 0);} \\
Causes the Node to print out its tag, mass matrix, and committed
response quantities. \\ 

\indent {\em int displaySelf(Renderer \&theRenderer, int flag,
float fact);} \\
Causes the Node to display itself. If {\em flag} is $1$ the Node will
cause its tag to be printed to the display. 



