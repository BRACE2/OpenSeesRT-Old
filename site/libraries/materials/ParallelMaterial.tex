%File: ~/OOP/material/ParallelModel.tex
%What: "@(#) ParallelModel.tex, revA"

UNDER CONSTRUCTION. POSSIBLE NAME CHANGE IF MATERIAL GENERAL.\\

\noindent {\bf Files}   \\
\#include $<\tilde{ }$/material/ParallelModel.h$>$  


\noindent {\bf Class Declaration}  \\
class ParallelModel: public MaterialModel 


\noindent {\bf Class Hierarchy} \\
TaggedObject 

MovableObject 

\indent\indent MaterialModel \\
\indent\indent\indent UniaxialMaterial \\
\indent\indent\indent\indent {\bf ParallelModel} \\

\noindent {\bf Description}  \\
\indent A ParallelModel object is an aggregation of
UniaxialMaterial objects all considered acting in parallel. SOME
THEORY. \\ 

\noindent {\bf Class Interface} \\
// Constructor 

\indent {\em ParallelModel(int tag, int numModel,
UniaxialMaterial **theModels);}  \\ \\
// Destructor 

{\em $\tilde{ }$ParallelModel();}\\ 

// Public Methods 

{\em int setTrialStrain(double strain); } 

{\em double getStress(void); } 

{\em double getTangent(void); } 

{\em int commitState(void); } 

{\em int revertToLastCommit(void); } 

{\em int revertToStart(void); } 

{\em UniaxialMaterial *getCopy(void); } \\ 

// Public Methods for Output

{\em    int sendSelf(int commitTag, Channel \&theChannel); }

\indent {\em    int recvSelf(int commitTag, Channel \&theChannel, 
		 FEM\_ObjectBroker \&theBroker); }\\
{\em    void Print(OPS_Stream \&s, int flag =0);} 



