% File: ~/domain/constraints/MP\_Constraint.tex 

\noindent {\bf Files}   \\
\#include $<\tilde{ }$/domain/constraints/MP\_Constraint.h$>$  


\noindent {\bf Class Declaration}  \\
class MP\_Constraint: public DomainComponent  


\noindent {\bf Class Hierarchy} \\
TaggedObject 

MovableObject 

\indent\indent DomainComponent \\
\indent\indent\indent {\bf MP\_Constraint} \\

\noindent {\bf Description}  \\
\indent An MP\_Constraint represents a multiple point constraint in the
domain. A multiple point constraint imposes a relationship between the
displacement for certain dof at two nodes in the model, typically called
the {\em retained} node and the {\em constrained} node: $U_c = C_{cr} U_r$


An MP\_Constraint is responsible for providing information on the
relationship between the dof, this is in the form of a constraint
Matrix, $C_{cr}$, and two ID objects, {\em retainedID} and {\em
constrainedID} indicating the dof's at the nodes 
represented by $C_{cr}$. For example, for the following constraint
imposing a relationship between the displacements at node $1$, the 
constrained node, with the displacements at node $2$, the retained
node in a problem where the x,y,z components are identified as the
0,1,2 degrees-of-freedom:


$$
$ u_{1,x} = 2 u_{2,x} + u_{2,z} $
$$


$$
$ u_{1,y} = 3 u_{2,z}$
$$


the constraint matrix is:
$$ C_{cr} =
\left[
\begin{array}{cc}
2 & 1  \\
0 & 3  \\
\end{array}
\right] 

$$

$$


{\em constrainedID} = $[0$ $1]$ and {\em retainedID} $= [0$ $2]$. \\

\noindent {\bf Class Interface}  \\
\indent\indent  // Constructors  \\
\indent\indent {\em MP\_Constraint(int tag, int nodeRetain, int nodeConstr,
Matrix \&constraint, \\
\indent\indent\indent\indent\indent\indent ID \&constrainedDOF, ID
\&retainedDOF);}\\ 
\indent\indent {\em MP\_Constraint(int tag, int nodeRetain, int nodeConstr, \\
\indent\indent\indent\indent\indent\indent ID \&constrainedDOF, ID
\&retainedDOF, int classTag);}\\
\indent\indent {\em MP\_Constraint(int classTag);}\\ \\
\indent\indent  // Destructor  \\
\indent\indent {\em virtual $\tilde{ }$ MP\_Constraint();} \\ \\
\indent\indent // Public Methods  \\
\indent\indent {\em virtual int getNodeRetained(void) const; } \\
\indent\indent {\em virtual int getNodeConstrained(void) const; }\\
\indent\indent {\em virtual const ID \& getConstrainedDOFs(void) const;}\\
\indent\indent {\em virtual const ID \& getRetainedDOFs(void) const;}\\
\indent\indent {\em virtual int applyConstraint(double timeStamp);} \\
\indent\indent {\em virtual bool isTimeVarying(void) const;} \\
\indent\indent {\em virtual const Matrix \&getConstraint(void) const;}\\\\
\indent\indent // Public Methods for Output \\
\indent\indent {\em virtual int sendSelf(int commitTag, Channel \&theChannel);} \\ 
\indent\indent {\em virtual int recvSelf(int commitTag, Channel \&theChannel, FEM\_ObjectBroker
\&theBroker);} \\ 
\indent\indent {\em virtual void Print(OPS_Stream \&s, int flag = 0);} \\

\noindent {\bf Constructors}  \\
\indent {\em MP\_Constraint(int tag, int nodeRetain, int nodeConstr,
Matrix \&constraint, \\
\indent\indent\indent\indent\indent\indent
 ID \&constrainedDOF, ID \&retainedDOF);}\\ 
To construct a multiple point constraint where the constrained node is
given by {\em nodeConstr}, the retained node by {\em nodeRetain}, the
{\em constrainedID} by {\em constrainedDOF}, the {\em retainedID} by
{\em retainedDOF} and $C_{cr}$ by {\em constraint}. 
The integers {\em tag} and  CNSTRNT\_TAG\_MP\_Constraint
are passed to the DomainComponent classes constructor. New Matrix and ID objects are
created to hold the information. \\

{\em MP\_Constraint(int tag, int nodeRetain, int nodeConstr, 

\indent\indent\indent\indent\indent\indent ID \&constrainedDOF, ID
\&retainedDOF, int classTag);}\\
For the subclasses to use. The subclasses can vary the contents of the
Matrix returned when {\em getConstraint()} is invoked. 
The integers {\em tag} {\em classTag} are
passed to the DomainComponent classes constructor. New ID objects are
created to hold the information. \\

{\em MP\_Constraint(int classTag);} 

Provided for the FEM\_ObjectBroker to construct a blank object. The
data for the object is filled in when {\em recvSelf()} is invoked on
the object. $0$ and {\em classTag} are passed to the DomainComponent
constructor. \\ 

\noindent {\bf Destructor}  \\
{\em virtual~$\tilde{}$ MP\_Constraint();} 

Invokes the destructor on both the ID and the Matrix object, if a
Matrix object is passed in the constructor. \\

\noindent {\bf Public Methods}  \\
{\em virtual int getNodeRetained(void) const; } 

Returns the value of {\em nodeRetain} passed in the constructor,
i.e. the tag of the retained node. \\

{\em virtual int getNodeConstrained(void) const;    }\\
Returns the value of {\em nodeConstr} passed in the constructor, i.e. the
tag of the constrained node. \\

{\em virtual const ID \&getConstrainedDOFs(void) const;   }\\     
Returns, as a const, the {\em constrainedID} formed in the constructor. \\

{\em virtual const ID \&getRetainedDOFs(void) const;   }\\     
Returns, as a const, the {\em retainedID} formed in the
constructor. \\

{\em virtual int applyConstraint(double timeStamp)}\\     
A method to invoked to inform the MP\_Constraint to determine $C_{cr}$,
for the time {\em timeStamp}. {\bf The base class will do nothing, as
Matrix is assumed to be constant.} \\

{\em virtual const Matrix \&getConstraint(void) const;} \\
Returns the current constraint Matrix, that determined in the last
call to {\em applyConstraint()}. For the MP\_Constraint class, $C_{cr}$
determined in the constructor is returned. \\

{\em virtual int sendSelf(int commitTag, Channel \&theChannel);} \\
Creates a Vector, stores the MP\_Constraints tag, nodeRetain,
nodeConstrained and value in the Vector, and sends the Vector to the
Channel using the objects own database tag and {\em commitTag}. It then
sends the {\em participatingDOF} ID and the {\em constraint}
Matrix, again using the objects database tag and {\em
commitTag}. Returns $0$ if successful, a negative number if the
Channel object, {\em theChannel}, failed to send the data. \\ 

{\em virtual int recvSelf(int commitTag, Channel \&theChannel,
FEM\_ObjectBroker \&theBroker);} \\ 
Creates a Vector, receives the Vector from the Channel using {\em
commitTag} and the objects database tag, and sets theMP\_Constraints
tag, nodeRetain, nodeConstrained from the the Vector. Creates a
Vector and a Matrix, and then receives the {\em participatingDOF} ID
and the {\em constraint} Matrix into these objects. Returns $0$ if
successful, a negative number if the Channel object, {\em
theChannel}, failed to receive the data.\\ 

{\em virtual void Print(OPS_Stream \&s, int flag = 0);} \\
Prints out the MP\_Constraints tag, then the tags of the constrained
and retained nodes, then the two ID's and finally the constraint Matrix.\\






