%File: ~/OOP/modelbuilder/PartitionedModelBuilder.tex
%What: "@(#) PartitionedModelBuilder.tex, revA"

\noindent {\bf Files}   \\
\#include $<\tilde{ }$/modelbuilder/PartitionedModelBuilder.h$>$  


\noindent {\bf Class Decleration}  \\
class PartitionedModelBuilder: public ModelBuilder, public MovableObject  


\noindent {\bf Class Hierarchy} \\
 PartitionedModelBuilder 

 MovableObject 

\indent\indent {\bf PartitionedModelBuilder} \\

\noindent {\bf Description} \\ 
\indent The PartitionedModelBuilder class is an abstract class. A subclass
of PartitionedModelBuilder is a class which creates a partitioned finite element
discretization of a structure: that is it discretizes the structure to
be modeled into Elements, Nodes, Constraints, etc. and adds these
components to the Subdomains of a PartitionedDomain.
PartitionedModelBuilders can be used for creating models for analysis
involving domain decomposition methods where there exist natural
partitions or where a model has previously been partitioned and this
information has been saved. \\

\noindent {\bf Class Interface} \\
\indent\indent // Constructors \\ 
\indent\indent {\em PartitionedModelBuilder(PartitionedDomain
\&theDomain, int classTag);}\\ 
\indent\indent {\em PartitionedModelBuilder(Subdomain \&aSubdomain, int
classTag);} \\ \\
\indent\indent {\em PartitionedModelBuilder(PartitionedDomain
\&theDomain, int classTag);}\\ \\
\indent\indent // Destructor \\ 
\indent\indent {\em virtual $\tilde{ }$PartitionedModelBuilder();}\\  \\
\indent\indent // Public Methods \\ 
\indent\indent {\em virtual buildFE\_Model(void);} \\
\indent\indent {\em virtual int buildInterface(int numSubdomains) =0;} \\ 
\indent\indent {\em virtual int buildSubdomain(int partition, int
numPartitions, Subdomain \&theSubdomain) =0;} \\ \\
\indent\indent // Protected Methods \\ 
\indent\indent {\em  PartitionedDomain *getPartitionedDomainPtr(void)
const;} \\

\noindent {\bf Constructors} \\ 
\indent {\em PartitionedModelBuilder(PartitionedDomain \&domain, int
classTag);}\\  
Typically, a PartitionedModelBuilder is associated with a PartitionedDomain,
this constructor sets up a link for the PartitionedModelBuilder and the domain,
setting its link to the Domain object {\em domain}. The Domain {\em
domain} is passed to the constructor for ModelBuilder, and the integer
{\em classTag} is passed to the MovableObject classes constructor. \\

\indent {\em PartitionedModelBuilder(Subomain \&domain, int classTag);}\\  
This is the constructor that is called when a PartitionedModelBuilder
is to be created by an FE\_ObjectBroker. The only method that can be
invoked on such an object is {\em buildSubdomain()}. \\

\noindent {\bf Destructor} \\
\indent {\em virtual~ $\tilde{}$PartitionedModelBuilder();}\\ 
Does nothing. \\


\noindent {\bf Public Methods} \\
{\em virtual buildFE\_Model(void);} 

The PartitionedModelBuilder will first check that the
PartitionedModelBuilder was constructed with a PartitioneDomain, if
not a warning message is printed and a $-1$ is returned. If o.k. the object 
then determines the number of Subdomains, {\em numSub} in the
PartitionedDomain. It then invokes {\em buildInterface(numSub)} on 
itself. Then for each Subdomain in the PartitionedDomain it invokes
{\em buildSubdomain(numSub, *this)}. If building the interface or any
of the subdomains fails, a warning message is printed and a negative
number is returned. Returns $0$ if sucessfull. \\

\indent {\em virtual int buildInterface(int numSubdomains) =0;}\\ 
This method must be provided by the subclasses. It is used to add any
boundary nodes, nodal loads and constraints to the PartitionedDomain
object. The integer {\em numSubdomains} is passed to provide
information about the number of subdomains in the
PartitionedDomain. To return $0$ if successfull, a negative number if not. \\


\indent {\em virtual int buildSubdomain(int partition, int
numPartitions, Subdomain \&theSubdomain) =0;} \\
This method must be provided by the subclasses. It is used to add
nodes, elements, loads and constraints to the subdomain {\em
theSubdomain}. The integers {\em partition} and {\em numPartitions}
are used to identify which partition is being built and the total
number of partitions. To return $0$ if succesfull, a negative number
if not. \\


\noindent {\bf Protected Methods} \\
{\em  PartitionedDomain *getPartitionedDomainPtr(void) const;} 

Returns a pointer to the PartitionedDomain object passed in the constructor.\\


