%File: ~/OOP/actor/machineBroker/MachineBroker.tex
%What: "@(#) MachineBroker.tex, revA"

\noindent {\bf Files}   \\
\#include $<\tilde{ }$/actor/machineBroker/MachineBroker.h$>$  


\noindent {\bf Class Declaration}  \\
class MachineBroker 


\noindent {\bf Class Hierarchy} \\
{\bf MachineBroker} 


\noindent {\bf Description}  \\
\indent MachineBrokers are objects that are used to start remote
processes running on the parallel machine. \\


\noindent {\bf Constructor}  \\
// Constructor  

{\em MachineBroker();}  \\

// Destructor 

{\em virtual $\tilde{ }$MachineBroker();}\\  

// Public Member Functions  

\indent {\em virtual int startActor(char *actorProgram, Channel \&theChannel,
int compDemand =0) =0;}\\

\noindent {\bf Constructor}  \\
{\em MachineBroker();}  


\noindent {\bf Destructor} \\
\indent {\em virtual~ $\tilde{}$MachineBroker();}\\ 
Does nothing. \\

\noindent {\bf Public Methods }  \\
\indent {\em virtual int startActor(char *actorProgram, Channel \&theChannel,
int compDemand =0) =0;}\\
Invoked to start the program, {\em actorProgram}, on the parallel
machine. The remote actor process uses information supplied by {\em
theChannel} to allow the remote process to connect to the local
process. The integer {\em compDemand} provides an indication of the
computational demands of the remote process in a heterogeneous
environment. 