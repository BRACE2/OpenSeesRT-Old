%File: ~/OOP/actor/Message.tex
%What: "@(#) Message.tex, revA"

\noindent {\bf Files}   \\
\#include $<\tilde{ }$/actor/message/Message.h$>$  


\noindent {\bf Class Declaration}  \\
class Message 


\noindent {\bf Class Hierarchy} \\
{\bf Message} 


\noindent {\bf Description}  \\
\indent Messages are objects that can be sent between Channels. They
are provided to allow data of arbitrary length and type, e.g. structs,
to be sent between processes running on similar machine
architectures. WARNING Sending Messages between machines with different
architectures can result in erroniuos data being received. Each
Message object keeps a pointer to the data it represents and an integer
outlining the data size. There is no copy of the data kept by the
Message. \\


\noindent {\bf Constructors}  \\
// Constructors  

{\em Message();}  

{\em Message(double *, int num);}

{\em Message(int *, int num);}

{\em Message(char *, int num);} \\ 

// Destructor 

{\em virtual~ $\tilde{}$Message();}\\  

// Public Member Functions  

\indent {\em virtual int putData(char *theData, int startLoc, int endLoc);}; \\  
{\em virtual const char *getData(void);}

{\em virtual int   getSize(void);} 


\noindent {\bf Constructors}  \\
{\em Message();}  

To construct an empty message. \\

{\em Message(double *data, int num);}

To construct a message for sending/receiving an array containing {\em
num} doubles. \\

{\em Message(int *data, int num);}

To construct a message for sending/receiving an array containing {\em
num} ints. \\

{\em Message(char *data, int num);} 

To construct a message for sending/receiving a string of {\em num}
characters or a struct. \\ 

\noindent {\bf Destructor} \\
\indent {\em virtual~ $\tilde{}$Message();}\\ 
Does nothing. \\

\noindent {\bf Public Methods }  \\
\indent {\em    virtual int putData(char *theData, int startLoc, int
endLoc);}; \\ 
A method which will put the data given by the character pointer {\em
theData} of size {\em endLoc -startLoc} into the data array pointed to
by the Message starting at location $startLoc$ in this array. Returns $0$ if
successful; an error message is printed and a $-1$ is returned if
not. The routine {\em bcopy()} is used to copy the data. \\ 

{\em  virtual const char *getData(void);}\\
A method which returns a const char * pointer to the messages data. \\

{\em  virtual int   getSize(void);} \\
A method to get the size of the array. The unit of size is that of a
character. 

