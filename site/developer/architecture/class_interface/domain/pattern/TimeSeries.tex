%File: ~/OOP/domain/pattern/TimeSeries.tex
%What: "@(#) TimeSeries.tex, revA"

\noindent {\bf Files}   \\
\#include $<\tilde{ }$domain/pattern/TimeSeries.h$>$  


\noindent {\bf Class Declaration}  \\
class TimeSeries: public DomainComponent  


\noindent {\bf Class Hierarchy} \\
MovableObject 

\indent\indent {\bf TimeSeries} \\

\noindent {\bf Description} \\ 
\indent The TimeSeries class is an abstract base class. A
TimeSeries object is used in a LoadPattern to determine the current
load factor to be applied to the loads and constraints for the time
specified. \\ 

\noindent {\bf Class Interface} \\
\indent // Constructor \\ 
{\em TimeSeries(int classTag);}\\ 

\indent // Destructor \\ 
{\em virtual $\tilde{ }$TimeSeries();}\\  

\indent // Pure Virtual Public Methods \\ 
{\em  virtual double getFactor(double pseudoTime) =0;}

{\em  virtual void Print(OPS_Stream \&s, int flag =0) =0;}


\noindent {\bf Constructor} \\ 
\indent {\em TimeSeries(int tag);}\\ 
The integer {\em classTag} is passed to the MovableObject classes
constructor. \\

\noindent {\bf Destructor} \\
\indent {\em virtual $\tilde{ }$TimeSeries();}\\ 
Does nothing. \\

\noindent {\bf Public Methods} \\
{\em  virtual double getFactor(double pseudoTime) =0;}

To return the current load factor for the given value of {\em
pseudoTime} to be applied to the loads and single-point constraints in
a LoadPattern based on the value of {\em pseudoTime}. \\

{\em  virtual void Print(OPS_Stream \&s, int flag =0) =0;}

To print to the stream {\em s} output based on the value of {\em flag}.