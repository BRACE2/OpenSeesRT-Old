%File: ~/OOP/analysis/integrator/CentralDifference.tex
%What: "@(#) CentralDifference.tex, revA"

\noindent {\bf Files}   \\
\#include $<\tilde{ }$/analysis/integrator/CentralDifference.h$>$  


\noindent {\bf Class Declaration}  \\
class CentralDifference: public TransientIntegrator  


\noindent {\bf Class Hierarchy} \\
MovableObject 

\indent\indent Integrator \\
\indent\indent\indent IncrementalIntegrator \\
\indent\indent\indent\indent TransientIntegrator \\
\indent\indent\indent\indent\indent {\bf CentralDifference} \\

\noindent {\bf Description} \\ 
\indent CentralDifference is a subclass of TransientIntegrator which implements
the CentralDifference method. In the CentralDifference method, to determine the
velocities, accelerations and displacements at time $t + \Delta t$,
the equilibrium equation (expressed for the TransientIntegrator) is
typically solved at time $t$ for $\U_{t+\Delta t}$, i.e. solve: 

\[ \R (\U_{t+ \Delta t}) = \P(t) - \F_I(\Udd_{t})
- \F_R(\Ud_{t},\U_{t}) \]

\noindent where we use following to relate $\Ud_{t}$ and $\Udd_{t}$ to
$\U_{t}$ and the displacement quantities at times $t$ and $t - \Delta
t$: 

\[
\dot \U_{t} = \frac{1}{2 \Delta t} \left(
\U_{t + \Delta t} -  \U_{t - \Delta t} \right)
\]

\[
\ddot \U_{t} = \frac{1}{\Delta t^2} \left(
\U_{t + \Delta t} - 2 \U_t + \U_{t - \Delta t}\right)
\]


\noindent which results in the following 

\[ \left[ \frac{1}{\Delta t^2} \M + \frac{1}{2 \Delta t}
\C \right] \U_{t + \Delta t} = \P(t) - F_I \left(\Udd_t^{(i-1)}
\right)
-F_R\left( \Ud_t^{(i-1)}, \U_t)\right) \]
\\

\noindent {\bf Class Interface} \\
\indent\indent // Constructors \\
\indent\indent {\em CentralDifference(bool dispFlag = true);}\\ 
\indent\indent {\em CentralDifference(double gamma, double beta, bool dispFlag = true);}\\ \\
\indent\indent // Destructor \\
\indent\indent {\em virtual~ $\tilde{}$CentralDifference();}\\ \\
\indent\indent // Public Methods \\
\indent\indent {\em int formEleTangent(FE\_Element *theEle);} \\
\indent\indent {\em int formNodTangent(DOF\_Group *theDof);} \\
\indent\indent {\em int domainChanged(void);}\\
\indent\indent {\em int newStep(double deltaT);}\\
\indent\indent {\em int update(const Vector \&$\Delta U$);} \\ \\
\indent\indent // Public Methods for Output\\
\indent\indent {\em int sendSelf(int commitTag, Channel \&theChannel);}\\ 
\indent\indent {\em int recvSelf(int commitTag, Channel \&theChannel,
FEM\_ObjectBroker \&theBroker);}\\ 
\indent\indent {\em int Print(OPS_Stream \&s, int flag = 0);}\\


\noindent {\bf Constructors} \\
\indent {\em CentralDifference(bool dispFlag = true);}\\ 
Sets $\gamma$ to $1/2$ and $\beta$ to $1/4$. Sets a flag indicating
whether the incremental solution is done in terms of displacement,
$\Delta \U$, if {\em dispFlag} is {\em true}, or  
acceleration, $\Delta \ddot \U$, if {\em dispFlag} is {\em false}. \\

\indent {\em CentralDifference(double gamma, double beta, bool dispFlag = true);}\\ 
Sets $\gamma$ to {\em gamma} and $\beta$ to {\em beta}. Sets a flag
indicating whether the incremental solution is done in terms of
displacement or acceleration to {\em dispFlag}. \\

\noindent {\bf Destructor} \\
\indent {\em virtual~ $\tilde{}$CentralDifference();}\\ 
Invokes the destructor on the Vector objects created. \\

\noindent {\bf Public Methods}\\
{\em int formEleTangent(FE\_Element *theEle);} 

This tangent for each FE\_Element is defined to be $\K_e = c1 \K + c2
\D + c3 \M$, where c1,c2 and c3 were determined in the last invocation
of the {\em newStep()} method.  The method returns $0$ after
performing the following operations:
\begin{tabbing}
while \= \+ while \= while \= \kill
theEle-$>$zeroTang() \\
theEle-$>$addKtoTang(c1) \\
theEle-$>$addCtoTang(c2) \\
theEle-$>$addMtoTang(c3) 
\end{tabbing}


{\em int formNodTangent(DOF\_Group *theDof);} \\
The method returns $0$ after performing the following operations:
\begin{tabbing}
while \= \+ while \= while \= \kill
theDof-$>$zeroUnbalance() \\
theDof-$>$addMtoTang(c3) 
\end{tabbing}


{\em int domainChanged(void);}\\
If the size of the LinearSOE has changed, the object deletes any old Vectors
created and then creates $6$ new Vector objects of size equal to {\em
theLinearSOE-$>$getNumEqn()}. There is a Vector object created to store
the current displacement, velocity and accelerations at times $t$ and
$t + \Delta t$. The response quantities at time $t + \Delta t$ are
then set by iterating over the DOF\_Group objects in the model and
obtaining their committed values. 
Returns $0$ if successful, otherwise a warning message and a negative
number is returned: $-1$ if no memory was available for constructing
the Vectors. \\

{\em int newStep(double $\Delta t$);}\\
The following are performed when this method is invoked:
\begin{enumerate}
\item First sets the values of the three constants {\em c1}, {\em c2}
and {\em c3} depending on the flag indicating whether incremental
displacements or accelerations are being solved for at each iteration.
If {\em dispFlag} was {\em true}, {\em c1} is set to $1.0$, {\em c2} to $
\gamma / (\beta * deltaT)$ and {\em c3} to $1/ (\beta * deltaT^2)$. If
the flag is {\em false} {\em c1} is set to $\beta * deltaT^2$, {\em c2} to $
\gamma * deltaT$ and {\em c3} to $1.0$. 
\item Then the Vectors for response quantities at time $t$ are set
equal to those at time $t + \Delta t$.
\item Then the velocity and accelerations approximations at time $t +
\delta t$ are set using the difference approximations if {\em
dispFlag} was {\em true}. (displacement and velocity if {\em false}).
\item The response quantities at the DOF\_Group objects are updated
with the new approximations by invoking {\em setResponse()} on the
AnalysisModel with new quantities for time $t + \Delta t$.
\item current time is obtained from the AnalysisModel, incremented by
$\Delta t$, and {\em applyLoad(time, 1.0)} is invoked on the
AnalysisModel. 
\item Finally {\em updateDomain()} is invoked on the AnalysisModel.
\end{enumerate}
The method returns $0$ if successful, otherwise a negative number is
returned: $-1$ if $\gamma$ or $\beta$ are $0$, $-2$ if {\em dispFlag}
was true and $\Delta t$ is $0$, and $-3$ if {\em domainChanged()}
failed or has not been called. \\

{\em int update(const Vector \&$\Delta U$);} \\
Invoked this causes the object to increment the DOF\_Group
response quantities at time $t + \Delta t$. The displacement Vector is  
incremented by $ c1 * \Delta U$, the velocity Vector by $
c2 * \Delta U$, and the acceleration Vector by $c3 * \Delta U$. 
The response at the DOF\_Group objects are then updated by invoking
{\em setResponse()} on the AnalysisModel with quantities at time $t +
\Delta t$. Finally {\em updateDomain()} is invoked on the
AnalysisModel. Returns
$0$ if successful. A warning message is printed and a negative number
returned if an error occurs: $-1$ if no associated AnalysisModel, $-2$
if the Vector objects have not been created, $-3$ if the Vector
objects and $\delta U$ are of different sizes. \\

{\em int sendSelf(int commitTag, Channel \&theChannel); } \\ 
Places the $\beta$ and $\gamma$ and {\em dispFlag} into a vector if size 3
and invokes {\em sendVector} on the Channel with this Vector. Returns
$0$ if successful, a warning message is printed and a $-1$ is
returned if {\em theChannel} fails to send the Vector. \\ 

{\em int recvSelf(int commitTag, Channel \&theChannel, 
FEM\_ObjectBroker \&theBroker); } \\ 
Receives in a Vector of size 3 the values of $\beta$, $\gamma$ and
{\em dispFlag}. Returns $0$ if successful, a warning message is printed,
$\delta \lambda$ is set to $0$, and a $-1$ is returned if {\em
theChannel} fails to receive the Vector.\\

{\em int Print(OPS_Stream \&s, int flag = 0);}\\
The object sends to $s$ its type, the current time, $\gamma$ and
$\beta$. 






