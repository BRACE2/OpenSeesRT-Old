%File: ~/OOP/actor/MovableObject.tex
%What: "@(#) MovableObject.tex, revA"

\noindent {\bf Files}   \\
\#include $<$/actor/actor/MovableObject.h$>$  


\noindent {\bf Class Declaration}  \\
class MovableObject 


\noindent {\bf Class Hierarchy} \\
{\bf MovableObject} 


\noindent {\bf Description}  \\
\indent MovableObject is an abstract class, i.e. no instances of MovableObject
should exist. MovableObjects are objects which are able to
send/receive themselves to/from Channel objects. With each movable
object is associated a unique class identifier, it is this id which
will allow object brokers in remote processes to create an object of
the correct type. In addition when databases are being used, each
MovableObject will have a unique database tag, it is this integer
which will allow the objects to retrieve their own data from the database. \\ 


\noindent {\bf Class Interface }  \\
\indent\indent // Constructor  \\
\indent\indent {\em MovableObject(int classTag, int dbTag);}  \\
\indent\indent {\em MovableObject(int classTag);}  \\ \\
\indent\indent // Destructor \\
\indent\indent {\em virtual~ $\tilde{}$MovableObject();}\\  \\
\indent\indent // Public Methods  \\
\indent\indent {\em int getClassTag(void) const;}\\
\indent\indent {\em int getDbTag(void) const;}\\
\indent\indent {\em void setDbTag(int dbTag);}\\
\indent\indent {\em virtual int sendSelf(int commitTag, Channel \&theChannel) =0;} \\ 
\indent\indent {\em virtual int recvSelf(int commitTag, Channel
\&theChannel, FEM\_ObjectBroker \&theBroker) =0;} \\ 


\noindent {\bf Constructor}  \\
{\em MovableObject(int classTag, int dbTag);}  

The constructor sets the objects class identifier to {\em classTag}:
this is a unique id for each class of instantiable movable
objects. The identifier will allow an object broker to recognize the
object type to be instantiated. Sets the objects database tag to {\em
dbTag}: this is a unique id for identifying the object in a database. \\


{\em MovableObject(int classTag);}  

The constructor sets the objects class identifier to {\em classTag} 
and sets the objects database tag to {\em 0}. \\

\noindent {\bf Destructor} \\
\indent {\em virtual~ $\tilde{}$MovableObject();}\\ 

\noindent {\bf Public Methods }  \\
{\em int getClassTag(void) const;}

A method which returns the class id, {\em classTag}, provided in
the constructor. \\

{\em int getDbTag(void) const;}\\
A method which returns the database tag, {\em dbTag}, provided in
the constructor or last set in {\em setDbTag()}. \\

{\em void setDbTag(int dbTag);}\\
A method to set the database tag to be equal to {\em dbTag}. \\

{\em virtual int sendSelf(int commitTag, Channel \&theChannel) =0;} \\ 
This is a pure virtual method, one must be written for each
instantiable subclass of MovableObject. Each object has to send the
data needed to be able to reproduce that object in a remote
process. The object uses the methods provided by {\em theChannel}
object to send the data to another channel at the remote actor, the
address of the channel is set before this method is called.
An object of similar type at the remote actor is invoked with a  {\em
receiveSelf()} to receive the data. Returns $0$ if successful
(successful in that the data got to the channel), or a $-1$ if no
data was sent. \\  

{\em virtual int recvSelf(int commitTag, Channel \&theChannel, FEM\_ObjectBroker
\&theBroker) =0;} \\ 
This is a pure virtual method, one must be written for each
instantiable subclass of MovableObject. Each object has to receive the
data needed to be able to recreate itself in the new process after it
has been sent through {\em theChannel}. If the object is an
aggregation containing other objects, new objects of the correct type
can be constructed using {\em theBroker}. To return $0$ if successful
or a $-1$ if not. \\




