%File: ~/database/FE\_Datastore.tex
%What: "@(#) FE\_Datastore.tex, revA"

\noindent {\bf Files}   \\
\#include $<\tilde{ }$/database/FE\_Datastore.h$>$  


\noindent {\bf Class Declaration}  \\
class FE\_Datastore: public ModelBuilder, public Channel 


\noindent {\bf Class Hierarchy} \\
ModelBuilder 

Channel 

\indent\indent {\bf FE\_Datastore} \\

\noindent {\bf Description}  \\
\indent FE\_Datastore is an abstract class. An FE\_Datastore object is
used in the program to store/restore the geometry and state information 
in the domain at particular instances. How, where and how the data is
stored depends on the implementation provided by the concrete
subclasses. \\  

\noindent {\bf Class Interface} \\
// Constructor 

{\em FE\_Datastore(Domain \&theDomain, FEM\_ObjectBroker \&theBroker);}\\ 

// Destructor 

{\em virtual~ $\tilde{}$FE\_Datastore();}\\ 

// Public Methods  

{\em virtual int getDbTag(void) =0;}

{\em virtual int commitState(int commitTag);}

{\em virtual int restoreState(int commitTag);}\\

{\em virtual int validateBaseRelationsWrite(int commitTag)=0;}

{\em virtual int validateBaseRelationsRead(int commitTag)=0;}\\ 

// Protected Methods  

{\em FEM\_ObjectBroker *getObjectBroker(void);}


\noindent {\bf Constructor}  \\
{\em FE\_Datastore(Domain \&theDomain);}  \\ 

The Domain object {\em theDomain} is passed to the ModelBuilder
constructor. A pointer is kept to {\em theBroker} object.\\

\noindent {\bf Destructor} \\
\indent {\em virtual~ $\tilde{}$FE\_Datastore();}\\ 
Does nothing. \\

\noindent {\bf Public Methods }  \\
{\em virtual int getDbTag(void) =0;}

To return a unique integer identifier at each call. This identifier
will be used by the objects to store/retrieve their information
to/from the database. \\

{\em virtual int commitState(int commitTag) = 0;}

Invoked to store the current state of the domain in the database. The
integer {\em commitTag} is used to identify the state for subsequent
calls to restore the information from the database. To return
$0$ if successful, a negative number if not. 

In the implementation for the FE\_Datastore class, the object first
invokes {\em validateBaseRelationsWrite()} on itself. If this method 
returns $0$, the object then loops over all the components of the
Domain object invoking {\em sendSelf(commitTag, this)} on each of
these objects. Returns $0$ if successful, a negative number if
not. For each domain component that could not send itself a warning
message is printed. \\ 

{\em virtual int restoreState(int commitTag) = 0;}

Invoked to restore the state of the domain from a database. The state
of the domain at the end of this call is to be the same as the state
of the domain when {\em commitState(commitTag)} was invoked. To return
$0$ if successful, a negative number if not. 

In the implementation for the FE\_Datastore class, the object first
invokes {\em validateBaseRelationsRead()} on itself. If this method 
returns $0$, the object then loops over all the components of the
Domain object invoking {\em recvSelf(commitTag, this)} on each of
these objects. Returns $0$ if successful, a negative number if
not. For each domain component that could not send itself a warning
message is printed. \\ 


{\em virtual int validateBaseRelationsWrite(int commitTag)=0;}

This method is invoked before the information can be sent to the
database. It is required to ensure that: \begin{enumerate} 
\item Each Node, Element, SP\_Constraint, MP\_Constraint, NodalLoad and
ElementalLoad which is to save information in the database has a
database tag.
\item That the information in the base tables is up to date so that a
later call to {\em validateBaseRelationsRead(commitTag)} will be successful.
\end{enumerate}
\noindent To return $0$ if the base relations are up to date, to return
$1$ if they are up to date and the component data has been sent to the
database, and a negative number if the method fails. \\

{\em virtual int validateBaseRelationsRead(int commitTag) =0;}

This method is invoked before the information can be extracted from the
database. It is required to ensure that the Domain has the same
type of DomainComponent objects and that each of these has the same
database tag as when {\em validateBaseRealationsWrie(commitTag)} was
invoked.  To return $0$ if the base relations are up to date, to return
$1$ if they are up to date and the component data has been received from the
database, and a negative number if the method fails. \\

{\em FEM\_ObjectBroker *getObjectBroker(void);}

Returns a pointer to {\em theBroker} object passed in the constructor. 