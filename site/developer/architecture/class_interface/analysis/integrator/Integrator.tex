%File: ~/OOP/analysis/integrator/Integrator.tex
%What: "@(#) Integrator.tex, revA"

\noindent {\bf Files}   \\
\#include $<\tilde{ }$/analysis/integrator/Integrator.h$>$  


\noindent {\bf Class Declaration}  \\
class Integrator: public MovableObject;  


\noindent {\bf Class Hierarchy} \\
MovableObject 

\indent\indent {\bf Integrator} \\

\noindent {\bf Description} \\ 
\indent The Integrator class is an abstract base class. Its purpose is
to define the interface common among all subclasses. An integrator
method is responsible for defining how the system of equations are set
up (this it does by specifying how the FE\_Element and DOF\_Group
objects of the analysis model construct the vectors and matrices asked
of them by the Analysis). It is also responsible for updating the
response quantities at the DOFs with the appropriate values; the values
are determined from the solution to the system of equations. \\

\noindent {\bf Class Interface} \\
\indent\indent // Constructor \\
\indent\indent {\em Integrator(int classTag);}\\ \\
\indent\indent // Destructor \\
\indent\indent {\em virtual $\tilde{ }$Integrator();}\\  \\
\indent\indent // Public Methods \\
\indent\indent {\em virtual int domainChanged(void); } \\
\indent\indent {\em  virtual int formEleTangent(FE\_Element *theEle) =0;} \\
\indent\indent {\em  virtual int formNodTangent(DOF\_Group *theDof) =0;} \\
\indent\indent {\em  virtual int formEleResidual(FE\_Element *theEle) =0;} \\
\indent\indent {\em  virtual int formNodUnbalance(DOF\_Group *theDof)
=0;} \\ \\
\indent\indent // Public Method added for Domain Decomposition \\
\indent\indent {\em virtual  int getLastResponse(Vector \&result, const
ID \&id) =0;} \\ \\
\indent\indent // Public Method added for Output \\
\indent\indent {\em virtual void Print(OPS_Stream \&s, int flag = 0) =0;}\\

\noindent {\bf Constructor} \\
\indent {\em Integrator(int classTag);}\\ 
{\em classTag} is passed to the MovableObject classes constructor. \\

\noindent {\bf Destructor} \\
\indent {\em virtual~$\tilde{}$Integrator();}\\ 
Does nothing. Provided so the subclass destructors will be invoked. \\

\noindent {\bf Public Methods}\\
{\em virtual int domainChanged(void); } 

Is called by the Analysis object. Refer to the Analysis classes to see
when it is invoked. To return $0$ if successful, a negative number if
not. This base class returns $0$.\\

{\em  virtual int formEleTangent(FE\_Element *theEle) =0;} 

Called upon to determine the FE\_Element {\em theEle}s matrix
contribution to the SystemOfEqn object. To return $0$ if successful,
a negative number otherwise. \\ 
 

{\em  virtual int formNodTangent(DOF\_Group *theDof) =0;} 

Called upon to determine the DOF\_Group {\em theDof}s matrix
contribution to the SystemOfEqn object. To return $0$ if successful,
a negative number otherwise. \\ 


{\em  virtual int formEleResidual(FE\_Element *theEle) =0;} 

Called upon to determine the FE\_Element {\em theEle}s vector
contribution to the SystemOfEqn object. To return $0$ if successful,
a negative number otherwise. \\ 


{\em virtual int formNodUnbalance(DOF\_Group *theDof) =0;} 

Called upon to determine the DOF\_Group {\em theDof}s vector
contribution to the SystemOfEqn object. To return $0$ if successful,
a negative number otherwise. \\ 

{\em virtual int getLastResponse(Vector \&result, const ID \&id) =0;} \\
Called upon to get the result quantities for the components specified
in the ID {\em id} and to place them in the Vector {\em result}. This
is provided for domain decomposition methods. To return $0$ if
successful, a negative number otherwise. \\ 

{\em virtual void Print(OPS_Stream \&s, int flag = 0) =0;}

The Integrator is to send information to the stream based on the
integer {\em flag}. 

