% File: ~/OOP/analysis/fe_ele/transformation/TransformationFE.tex 
%What: "@(#) TransformationFE.tex, revA"

\noindent {\bf Files}   \\
\#include $<\tilde{ }$/analysis/fe\_ele/penalty/TransformationFE.h$>$  


UNDER CONSTRUCTION. \\

\noindent {\bf Class Declaration}  \\
class TransformationFE: public FE\_Element ;  


\noindent {\bf Class Hierarchy}  \\
FE\_Element 

\indent\indent {\bf TransformationFE} \\ 

\noindent {\bf Description}  \\
\indent TransformationFE is a subclass of FE\_Element used to enforce a
multi point constraint, of the form $\U_c = \C_{cr} \U_r$, where $\U_c$ are
the constrained degrees-of-freedom at the constrained node, $\U_r$ are
the retained degrees-of-freedom at the retained node and $\C_{cr}$ a
matrix defining the relationship between these degrees-of-freedom. 

To enforce the constraint a matrix $\T^T \K \T$ is added to the
tangent and $\T^T \R$ is added to the residual where $\T$ is a block
diagonal matrix equal to WHAT?


\noindent {\bf Class Interface}  \\
// Constructor  

\indent {\em TransformationFE(Domain \&theDomain,
TransformationConstraintHandler \&theHandler); \\ \\
// Destructor  

{\em virtual~ $\tilde{}$TransformationFE();}  \\ 

// Public Methods 


{\em    virtual const ID \&getDOFtags(void) const; }

{\em   virtual const ID \&getID(void) const;}

{\em    void setAnalysisModel(AnalysisModel \&theModel);}

\indent {\em virtual void setID(void);} \\ 
\indent {\em virtual const Matrix \&getTangent(Integrator
*theIntegrator);} \\  
\indent {\em virtual const Vector \&getResidual(Integrator
*theIntegrator);} \\ 
\indent {\em virtual const Vector \&getTangForce(const Vector
\&disp, double fact = 1.0);    }\\
{\em    int addSP(SP\_Constraint \&theSP);}\\ 

// Protected Methods 

\indent {\em transformResponse(const Vector \&modResponse, 
Vector \&unmodResponse);}\\

\noindent {\bf Constructor}  \\
\indent {\em TransformationFE(Domain \&theDomain,
TransformationConstraintHandler \&theHandler); \\


\noindent {\bf Destructor}  \\
{\em virtual~ $\tilde{}$TransformationFE();}  



\noindent {\bf Public Methods}  \\
{\em virtual void setID(void);} 


{\em virtual Matrix \&getTangent(Integrator *theIntegrator);} 


{\em virtual const Vector \&getResidual(Integrator *theIntegrator);} 



{\em virtual const Vector \&getTangForce(const Vector \&disp, double
fact = 1.0);    }\\


