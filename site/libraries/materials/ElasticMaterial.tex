%File: ~/OOP/material/ElasticMaterial.tex
%What: "@(#) ElasticMaterial.tex, revA"

\noindent {\bf Files}   \\
\#include $<\tilde{ }$/material/ElasticMaterial.h$>$  


\noindent {\bf Class Declaration}  \\
class ElasticMaterial: public MaterialModel 


\noindent {\bf Class Hierarchy} \\
TaggedObject 

MovableObject 

\indent\indent MaterialModel \\
\indent\indent\indent UniaxialMaterial \\
\indent\indent\indent\indent {\bf ElasticMaterial} \\

\noindent {\bf Description}  \\
\indent ElasticMaterial provides the abstraction of an elastic
uniaxial material, i.e. the stress-strain relationship is given by the
linear equation $\sigma = E * \epsilon$. \\

\noindent {\bf Class Interface} \\
// Constructor 

{\em ElasticMaterial(int tag, double E);}  \\ 

// Destructor 

{\em $\tilde{ }$ElasticMaterial();}\\ 

// Public Methods 

{\em int setTrialStrain(double strain); } 

{\em double getStress(void); } 

{\em double getTangent(void); } 

{\em int commitState(void); } 

{\em int revertToLastCommit(void); } 

{\em int revertToStart(void); } 

{\em UniaxialMaterial *getCopy(void); } \\ 

// Public Methods for Output

{\em int sendSelf(int commitTag, Channel \&theChannel); }

\indent {\em int recvSelf(int commitTag, Channel \&theChannel, 
FEM\_ObjectBroker \&theBroker); }\\
{\em void Print(OPS_Stream \&s, int flag =0);} 


\noindent {\bf Constructor}  \\
{\em ElasticMaterial(int tag, int classTag);}  

To construct an ElasticMaterial with an integer identifier {\em
tag}, an elastic tangent modulus of $E$ and a current strain
$\epsilon$ of $0.0$. The integers {\em tag} and
MAT\_TAG\_ElasticMaterial, defined in $<$classTags.h$>$, are passed 
to the UniaxialMaterial classes constructor. \\

\noindent {\bf Destructor} \\
\indent {\em virtual~ $\tilde{}$ElasticMaterial();}\\ 
Does nothing. \\

\noindent {\bf Public Methods} \\
{\em int setTrialStrain(double strain); }  

Sets the value of the trial strain, $\epsilon$ to be {\em
strain}. Returns $0$.\\

{\em double getStress(void); } 

Returns the product of $E * \epsilon$, where $\epsilon$ is the current
trial strain. \\

{\em double getTangent(void); } 

Returns the value of $E$ passed in the constructor. \\

{\em int commitState(void); } 

Returns $0$. \\

{\em int revertToLastCommit(void); } 

Returns $0$. \\

{\em int revertToStart(void); } 

Returns $0$. \\

{\em UniaxialMaterial *getCopy(void); } 

returns a pointer to a new ElasticMaterial object, constructed
using the same values of {\em tag} and $E$. It is up to the caller to
ensure that the destructor is invoked. \\

{\em    int sendSelf(int commitTag, Channel \&theChannel); }

Creates a Vector of size $2$ into which it places  {\em tag} and
{\em E}. Invokes {\em sendVector()} on  {\em theChannel} using the
ElasticMaterialObjects {\em dbTag}, the integer {\em commitTag} and
the Vector as arguments. Returns $0$ if successful, a warning message
and a negative number are returned if the Channel object fails to send
the Vector. \\

\indent {\em    int recvSelf(int commitTag, Channel \&theChannel, 
		 FEM\_ObjectBroker \&theBroker); }\\
Creates a Vector of size $2$. Invokes {\em recvVector()} on  {\em
theChannel} using the ElasticMaterialObjects {\em dbTag}, the integer
{\em commitTag} and the Vector as arguments. Using the data in the
Vector to set it's {\em tag} and $E$. Returns $0$ if successful, a
warning message is printed, {\em tag} and $E$ are set to $0.0$, and a
negative number is returned if the Channel object fails to receive
the Vector. \\

{\em    void Print(OPS_Stream \&s, int flag =0);} 

Prints to the stream {\em s} the objects {\em tag} and $E$ values. \\
