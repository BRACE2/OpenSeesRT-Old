%File: ~/OOP/system_of_eqn/linearSOE/profileSPD/ProfileSPDLinSOE.tex
%What: "@(#) ProfileSPDLinSOE.tex, revA"

\noindent {\bf ProfileSPDLinSOE} \\

\noindent {\bf Files}   \\
\#include $<\tilde{ }$ProfileSPDLinSOE.h$>$  


\noindent {\bf Class Declaration}  \\
class ProfileSPDLinSOE: public LinearSOE 


\noindent {\bf Class Hierarchy} \\
MovableObject 

\indent\indent SystemOfEqn \\
\indent\indent\indent LinearSOE \\
\indent\indent\indent\indent {\bf ProfileSPDLinSOE} \\

\noindent {\bf Description}  \\
\indent ProfileSPDLinSOE is class which is used to store a symmetric
system of equations using a profile storage scheme. The upper
triangular part of $A$ is stored in a 1d double array with the diagonals of
$A$ located at positions given by an integer array $iLoc$. 
For example when $n=5$ and $A$ as shown below:


$$

$$

\left[
\begin{array}{ccccc}
a_{0,0} & a_{0,1}  & 0 & 0 & a_{0,4} \\
a_{1,0} & a_{1,1} & a_{1,2} & a_{1,3} & 0 \\
a_{2,0} & a_{2,1} & a_{2,2} & a_{2,3} & a_{2,4}  \\
0 & a_{3,1} & a_{3,2} & a_{3,3} & a_{3,4} \\
0 & 0 & a_{4,2} & a_{4,3} & a_{4,4} \\
\end{array}
\right] 

$$

$$


this is stored using:


$$ A =
\left[
\begin{array}{cccccccccccccccccccc}
a_{0,0} & a_{0,1}  & a_{1,1} & a_{1,2} & a_{2,2} & a_{1,3} &
a_{2,3} & a_{3,3} & a_{0,4} & 0 & a_{2,4} & a_{3,4} & a_{4,4}\\
\end{array}
\right] 

$$

$$


and 

$$ iLoc =
\left[
\begin{array}{cccccccccccccccccccc}
1 & 3 & 5 & 8 & 13 \\
\end{array}
\right] 

$$

$$

Note $iLoc$ stores the diagonal locations using Fortran indexing. This
is to facilitate calls to Fortran libraries, e.g. Digital's DXML.
The $x$ and $b$ vectors are stored in 1d double arrays of length $N$. \\


\noindent {\bf Interface}  \\
\indent\indent {// Constructors} \\
\indent\indent {\em ProfileSPDLinSOE(Solver \&theSolver);}  \\
\indent\indent {\em ProfileSPDLinSOE(int N, int *iLoc,
		  ProfileSPDLinSolver \&theSolver);}\\ \\
\indent\indent {// Destructor} \\
\indent\indent {\em  $\tilde{}$ProfileSPDLinSOE();}\\ \\
\indent\indent {// Public Methods }  \\
\indent\indent {\em  int setProfileSPDSolver(ProfileSPDLinSolver \&newSolver);}\\
\indent\indent {\em int setSize(const Graph \&theGraph) =0; } \\
\indent\indent {\em int getNumEqn(void) =0; } \\
\indent\indent {\em int addA(const Matrix \&theMatrix, const ID \& loc,
doublefact = 1.0) =0;} \\
\indent\indent {\em int addB(const Vector \& theVector, const ID \& loc,
double fact = 1.0) =0;} \\
\indent\indent {\em int setB(const Vector \& theVector, 
double fact = 1.0) =0;} \\
\indent\indent {\em void zeroA(void) =0;} \\
\indent\indent {\em void zeroB(void) =0;} \\
\indent\indent {\em const Vector \&getX(void) = 0;} \\
\indent\indent {\em const Vector \&getB(void) = 0;} \\
\indent\indent {\em double normRHS(void) =0;} \\
\indent\indent {\em void setX(int loc, double value) =0;}\\
\indent\indent {\em int sendSelf(int commitTag, Channel \&theChannel);}\\ 
\indent\indent {\em int recvSelf(int commitTag, Channel \&theChannel,
FEM\_ObjectBroker \&theBroker);}\\ 

\noindent {\bf Constructors}  \\
{\em ProfileSPDLinSOE(Solver \&theSolver);}  

The {\em solver} and a unique class tag (defined in $<$classTags.h$>$)
are passed to the LinearSOE constructor. The system size is set
to $0$ and the matrix $A$ is marked as not having been factored. Invokes
{\em setLinearSOE(*this)} on the {\em solver}. No memory is
allocated for the 1d arrays. \\  


{\em ProfileSPDLinSOE(int N, int *newIloc,
		  ProfileSPDLinSolver \&theSolver); }\\
The {\em solver} and a unique class tag (defined in $<$classTags.h$>$)
are passed to the LinearSOE constructor. The system size is set
to $N$ and the matrix $A$ is marked as not having been
factored or condensed. Obtains memory from the heap for the 1d arrays storing the
data for $A$, $x$, $b$ and $iLoc$ and stores the size of these arrays. If not
enough memory is available for these arrays a warning message is
printed and the system size is set to $0$. The size of $A$ is given
by $newIloc(N-1)$, if this is not a valid address in {\em newIloc} a
segmentation fault or erronious results will result. The contents of
$iLoc$ are set equal to those of {\em newIloc}. Invokes {\em
setLinearSOE(*this)} and {\em setSize()} on {\em solver},
printing a warning message if {\em setSize()} returns a negative
number. Also creates Vector objects for $x$ and $b$ using the {\em
(double *,int)} Vector constructor. \\

\noindent {\bf Destructor} \\
\indent {\em virtual~ $\tilde{}$ProfileSPDLinSOE();}\\ 
Calls delete on any arrays created. \\

\noindent {\bf Public Methods }  \\
{\em  int setProfileSPDSolver(ProfileSPDLinSolver \&newSolver);}

Invokes {\em setLinearSOE(*this)} on {\em newSolver}.
If the system size is not equal to $0$, it also invokes {\em setSize()}
on {\em newSolver}, printing a warning and returning the returned value if this
method returns a number less than $0$. Finally it returns the result
of invoking the LinearSOE classes {\em setSolver()} method. \\

{\em int getNumEqn(void) =0; } 

A method which returns the current size of the system. \\

\indent {\em  int setSize(const Graph \&G); } \\ 
The size of the system is determined by looking at the adjacency ID of
each Vertex in the Graph object {\em G}. This is done by first
determining the column height for each Vertex $i$ in {\em G}, done by
setting $iLoc(i)$ equal to $0$ and then checking for each Vertex
in {\em G}, whether any of the vertex tags in the Vertices adjacency
ID results in $iLoc(i)$ being increased. Knowing the col height of
each column, the values of {\em iLoc} can be determined. Knowing {\em
iLoc} and the size of the system (the number of Vertices in {\em G}, 
a check to see if the previously allocated 1d arrays for $A$, $x$ and
$b$ are large enough. If the memory portions allocated for the 1d
arrays are not big enough, the old space is returned to the heap and
new space is allocated from the heap. Printins a warning message if
not enough memory is available on the heap for the 1d arrays and
returns a $-1$. If memory is available, the components of the arrays
are zeroed and $A$ is marked as being unfactored. If the system size
has increased, new Vector objects for $x$ and $b$ using the {\em
(double *,int)} Vector constructor are created. Finally, the result of 
invoking {\em setSize()} on the associated Solver object is
returned. \\ 


\indent {\em int addA(const Matrix \&M, const ID \& loc,
doublefact = 1.0) =0;} \\
First tests that {\em loc} and {\em M} are of compatible sizes; if not
a warning message is printed and a $-1$ is returned. The LinearSOE
object then assembles {\em fact} times the Matrix {\em 
M} into the matrix $A$. The Matrix is assembled into $A$ at the
locations given by the ID object {\em loc}, i.e. $a_{loc(i),loc(j)} +=
fact * M(i,j)$. If the location specified is outside the range,
i.e. $(-1,-1)$ the corrseponding entry in {\em M} is not added to
$A$. If {\em fact} is equal to $0.0$ or $1.0$, more efficient steps
are performed. Returns $0$.  \\


{\em int addB(const Vector \& V, const ID \& loc,
double fact = 1.0) =0;} \\
First tests that {\em loc} and {\em V} are of compatible sizes; if not
a warning message is printed and a $-1$ is returned. The LinearSOE
object then assembles {\em fact} times the Vector {\em V} into
the vector $b$. The Vector is assembled into $b$ at the locations
given by the ID object {\em loc}, i.e. $b_{loc(i)} += fact * V(i)$. If a
location specified is outside the range, e.g. $-1$, the corresponding
entry in {\em V} is not added to $b$. If {\em fact} is equal to $0.0$,
$1.0$ or $-1.0$, more efficient steps are performed. Returns $0$. \\


{\em int setB(const Vector \& V, double fact = 1.0) =0;} \\
First tests that {\em V} and the size of the system are of compatible
sizes; if not a warning message is printed and a $-1$ is returned. The
LinearSOE object then sets the vector {\em b} to be {\em fact} times
the Vector {\em V}. If {\em fact} is equal to $0.0$, $1.0$ or $-1.0$,
more efficient steps are performed. Returns $0$. \\ 

{\em void zeroA(void) =0;} \\
Zeros the entries in the 1d array for $A$ and marks the system as not
having been factored. \\

{\em void zeroB(void) =0;} \\
Zeros the entries in the 1d array for $b$. \\

{\em const Vector \&getX(void) = 0;} \\
Returns the Vector object created for $x$. \\

{\em const Vector \&getB(void) = 0;} \\
Returns the Vector object created for $b$. \\

{\em double normRHS(void) =0;} \\
Returns the 2-norm of the vector $x$. \\

{\em void setX(int loc, double value) =0;}\\
If {\em loc} is within the range of $x$, sets $x(loc) = value$. \\

\indent {\em int sendSelf(int commitTag, Channel \&theChannel);}\\ 
Returns $0$. The object does not send any data or connectivity
information as this is not needed in the finite element design. \\

\indent {\em int recvSelf(int commitTag, Channel \&theChannel, FEM\_ObjectBroker
\&theBroker);}\\ 
Returns $0$. The object does not receive any data or connectivity
information as this is not needed in the finite element design.




