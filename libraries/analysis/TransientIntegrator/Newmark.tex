%File: ~/OOP/analysis/integrator/Newmark.tex
%What: "@(#) Newmark.tex, revA"

\noindent {\bf Files}   \\
\#include $<\tilde{ }$/analysis/integrator/Newmark.h$>$  


\noindent {\bf Class Declaration}  \\
class Newmark: public TransientIntegrator  


\noindent {\bf Class Hierarchy} \\
MovableObject 

\indent\indent Integrator \\
\indent\indent\indent IncrementalIntegrator \\
\indent\indent\indent\indent TransientIntegrator \\
\indent\indent\indent\indent\indent {\bf Newmark} \\

\noindent {\bf Description} \\ 
\indent Newmark is a subclass of TransientIntegrator which implements
the Newmark method. In the Newmark method, to determine the
velocities, accelerations and displacements at time $t + \Delta t$,
the equilibrium equation (expressed for the TransientIntegrator) is
typically solved at time $t + \Delta t$ for $\U_{t+\Delta t}$,
i.e. solve: 

$$ \R (\U_{t + \Delta t}) = \P(t + \Delta t) - \F_I(\Udd_{t+\Delta t})
- \F_R(\Ud_{t + \Delta t},\U_{t + \Delta t}) $$


\noindent for $\U_{t+\Delta t}$. The following difference relations
are used to relate $\Ud_{t + \Delta t}$ and $\Udd_{t + \Delta t}$ to
$\U_{t + \Delta t}$ and the response quantities at time $t$:

$$
\dot \U_{t + \Delta t} = \frac{\gamma}{\beta \Delta t}
\left( \U_{t + \Delta t} - \U_t \right)
 + \left( 1 - \frac{\gamma}{\beta}\right) \dot \U_t + \Delta t \left(1
- \frac{\gamma}{2 \beta}\right) \ddot \U_t 
$$


$$
\ddot \U_{t + \Delta t} = \frac{1}{\beta {\Delta t}^2}
\left( \U_{t+\Delta t} - \U_t \right)
 - \frac{1}{\beta \Delta t} \dot \U_t + \left( 1 - \frac{1}{2
\beta} \right) \ddot \U_t 
$$


\noindent which  results in the following 

$$ \left[ \frac{1}{\beta \Delta t^2} \M + \frac{\gamma}{\beta \Delta t}
\C + \K \right] \Delta \U_{t + \Delta t}^{(i)} = \P(t + \Delta t) -
\F_I\left(\Udd_{t+\Delta  t}^{(i-1)}\right)
- \F_R\left(\Ud_{t + \Delta t}^{(i-1)},\U_{t + \Delta t}^{(i-1)}\right) $$


\noindent An alternative approach, which does not involve $\Delta t$
in the denumerator (useful for impulse problems), is to solve for the
accelerations at time $t + \Delta t$ 

$$ \R (\Udd_{t + \Delta t}) = \P(t + \Delta t) - \F_I(\Udd_{t+\Delta t})
- \F_R(\Ud_{t + \Delta t},\U_{t + \Delta t}) $$


\noindent where we use following functions to relate $\U_{t + \Delta
t}$ and $\Ud_{t + \Delta t}$ to $\Udd_{t + \Delta t}$ and the response
quantities at time $t$:

$$
\U_{t + \Delta t} = \U_t + \Delta t \Ud_t + \left[
\left(\frac{1}{2} - \beta\right)\Udd_t + \beta \Udd_{t + \Delta
t}\right] \Delta t^2
$$


$$
\Ud_{t + \Delta t} = \Ud_t + \left[ \left(1 - \gamma\right)\Udd_t +
\gamma\Udd_{t + \Delta t}\right] \Delta t
$$


\noindent which results in the following 

$$ \left[ \M + \gamma \Delta t \C + \beta \Delta t^2 \K \right] \Delta
\Udd_{t + \Delta t}^{(i)} = \P(t + \Delta t) - \F_I\left(\Udd_{t+\Delta 
t}^{(i-1)}\right)
- \F_R\left(\Ud_{t + \Delta t}^{(i-1)},\U_{t + \Delta
t}^{(i-1)}\right) $$



\pagebreak
\noindent {\bf Class Interface} \\
// Constructors 

\indent {\em Newmark(bool dispFlag = true);}\\ 
\indent {\em Newmark(double gamma, double beta, bool dispFlag = true);}\\ 
\indent {\em Newmark(double gamma, double beta, double alphaM, double
betaK, bool dispFlag = true);}\\ \\
// Destructor 

{\em virtual~ $\tilde{}$Newmark();}\\ 

// Public Methods 

{\em int formEleTangent(FE\_Element *theEle);} 

{\em int formNodTangent(DOF\_Group *theDof);} 

{\em int domainChanged(void);}

{\em int newStep(double deltaT);}

{\em int update(const Vector \&$\Delta U$);} \\ 

// Public Methods for Output

\indent {\em int sendSelf(int commitTag, Channel \&theChannel);}\\ 
\indent {\em int recvSelf(int commitTag, Channel \&theChannel,
FEM\_ObjectBroker \&theBroker);}\\ 
{\em int Print(OPS_Stream \&s, int flag = 0);}



\noindent {\bf Constructors} \\
\indent {\em Newmark(bool dispFlag = true);}\\ 
Sets $\gamma$ to $1/2$ and $\beta$ to $1/4$. Sets a flag indicating
whether the incremental solution is done in terms of displacement,
$\Delta \U$, if {\em dispFlag} is {\em true}, or  
acceleration, $\Delta \ddot \U$, if {\em dispFlag} is {\em false}. In
addition, a flag is set indicating that Rayleigh damping will not be used. \\


\indent {\em Newmark(double gamma, double beta, bool dispFlag = true);}\\ 
Sets $\gamma$ to {\em gamma} and $\beta$ to {\em beta}. Sets a flag
indicating whether the incremental solution is done in terms of
displacement or acceleration to {\em dispFlag} and a flag indicating
that Rayleigh damping will not be used. \\


\indent {\em Newmark(double gamma, double beta, double alphaM, double
betaK, bool dispFlag = true);}\\ 
This constructor is invoked if Rayleigh damping is to be used, 
i.e. $\D = \alpha_M M + \beta_K K$. 
Sets $\gamma$ to {\em gamma}, $\beta$ to {\em beta}, $\alpha_M$ to
{\em alphaM} and $\beta_K$ to {\em betaK}. Sets a flag indicating whether the
incremental solution is done in terms of displacement or acceleration
to {\em dispFlag} and a flag indicating that Rayleigh damping will 
be used. \\ 

\noindent {\bf Destructor} \\
\indent {\em virtual~ $\tilde{}$Newmark();}\\ 
Invokes the destructor on the Vector objects created. \\

\noindent {\bf Public Methods}\\
{\em int formEleTangent(FE\_Element *theEle);} 

This tangent for each FE\_Element is defined to be $\K_e = c1 \K + c2
\D + c3 \M$, where c1,c2 and c3 were determined in the last invocation
of the {\em newStep()} method.  The method returns $0$ after
performing the following operations:
\begin{tabbing}
while \= \+ while \= while \= \kill
if (RayleighDamping == false) \{ \+ \\
theEle-$>$zeroTang() \\
theEle-$>$addKtoTang(c1) \\
theEle-$>$addCtoTang(c2) \\
theEle-$>$addMtoTang(c3)  \- \\
\} else \{ \+ \\
theEle-$>$zeroTang() \\
theEle-$>$addKtoTang(c1 + c2 * $\beta_K$) \\
theEle-$>$addMtoTang(c3 + c2 * $\alpha_M$)  \- \\ 
\}
\end{tabbing}



{\em int formNodTangent(DOF\_Group *theDof);} \\
The method returns $0$ after performing the following operations:
\begin{tabbing}
while \= \+ while \= while \= \kill
theDof-$>$zeroUnbalance() \\
if (RayleighDamping == false)  \+ \\
theDof-$>$addMtoTang(c3)  \- \\
else \+ \\
theDof-$>$addMtoTang(c3 + c2 * $\alpha_M$)  \- \\ 
\end{tabbing}


{\em int domainChanged(void);}\\
If the size of the LinearSOE has changed, the object deletes any old Vectors
created and then creates $6$ new Vector objects of size equal to {\em
theLinearSOE-$>$getNumEqn()}. There is a Vector object created to store
the current displacement, velocity and accelerations at times $t$ and
$t + \Delta t$. The response quantities at time $t + \Delta t$ are
then set by iterating over the DOF\_Group objects in the model and
obtaining their committed values. 
Returns $0$ if successful, otherwise a warning message and a negative
number is returned: $-1$ if no memory was available for constructing
the Vectors. \\

{\em int newStep(double $\Delta t$);}\\
The following are performed when this method is invoked:
\begin{enumerate}
\item First sets the values of the three constants {\em c1}, {\em c2}
and {\em c3} depending on the flag indicating whether incremental
displacements or accelerations are being solved for at each iteration.
If {\em dispFlag} was {\em true}, {\em c1} is set to $1.0$, {\em c2} to $
\gamma / (\beta \Delta t)$ and {\em c3} to $1/ (\beta \Delta t^2)$. If
the flag is {\em false} {\em c1} is set to $\beta \Delta t^2$, {\em c2} to $
\gamma \Delta t$ and {\em c3} to $1.0$. 
\item Then the Vectors for response quantities at time $t$ are set
equal to those at time $t + \Delta t$.
\begin{tabbing}
while \= while \= while \= while \= \kill
\>\> $ \U_t = \U_{t + \Delta t}$ \\
\>\> $ \Ud_t = \Ud_{t + \Delta t} $ \\
\>\> $ \Udd_t = \Udd_{t + \Delta t} $ 
\end{tabbing}
\item Then the velocity and accelerations approximations at time $t +
\Delta t$ are set using the difference approximations if {\em
dispFlag} was {\em true}. (displacement and velocity if {\em false}).
\begin{tabbing}
while \= while \= while \= while \= \kill
\>\> if (displIncr == true) \{ \\
\>\>\> $ \dot \U_{t + \Delta t} = 
 \left( 1 - \frac{\gamma}{\beta}\right) \dot \U_t + \Delta t \left(1
- \frac{\gamma}{2 \beta}\right) \ddot \U_t $ \\
\>\>\> $ \ddot \U_{t + \Delta t} = 
 - \frac{1}{\beta \Delta t} \dot \U_t + \left( 1 - \frac{1}{2
\beta} \right) \ddot \U_t  $ \\
\>\> \} else \{ \\
\>\>\> $ \U_{t + \Delta t} = \U_t + \Delta t \Ud_t + \frac{\Delta
t^2}{2}\Udd_t$ \\
\>\>\> $ \Ud_{t + \Delta t} = \Ud_t +  \Delta t \Udd_t $ \\
\>\> \} 
\end{tabbing}
\item The response quantities at the DOF\_Group objects are updated
with the new approximations by invoking {\em setResponse()} on the
AnalysisModel with new quantities for time $t + \Delta t$.
\begin{tabbing}
while \= while \= while \= while \= \kill
\>\> theModel-$>$setResponse$(\U_{t + \Delta t}, \Ud_{t+\Delta t},
\Udd_{t+\Delta t})$ 
\end{tabbing}
\item current time is obtained from the AnalysisModel, incremented by
$\Delta t$, and {\em applyLoad(time, 1.0)} is invoked on the
AnalysisModel. 
\item Finally {\em updateDomain()} is invoked on the AnalysisModel.
\end{enumerate}
The method returns $0$ if successful, otherwise a negative number is
returned: $-1$ if $\gamma$ or $\beta$ are $0$, $-2$ if {\em dispFlag}
was true and $\Delta t$ is $0$, and $-3$ if {\em domainChanged()}
failed or has not been called. \\

{\em int update(const Vector \&$\Delta U$);} \\
Invoked this causes the object to increment the DOF\_Group
response quantities at time $t + \Delta t$. The displacement Vector is  
incremented by $ c1 * \Delta U$, the velocity Vector by $
c2 * \Delta U$, and the acceleration Vector by $c3 * \Delta U$.
The response at the DOF\_Group objects are then updated by invoking
{\em setResponse()} on the AnalysisModel with quantities at time $t +
\Delta t$. Finally {\em updateDomain()} is invoked on the 
AnalysisModel. 
\begin{tabbing}
while \= while \= while \= while \= \kill
\>\> if (displIncr == true) \{ \\
\>\>\> $ \U_{t + \Delta t} += \Delta \U$ \\
\>\>\> $ \dot \U_{t + \Delta t} += \frac{\gamma}{\beta \Delta t} \Delta \U $\\
\>\>\> $ \ddot \U_{t + \Delta t} += \frac{1}{\beta {\Delta t}^2} \Delta
\U $\\
\>\> \} else \{ \\
\>\>\> $ \Udd_{t + \Delta t} += \Delta \Udd$ \\
\>\>\> $ \U_{t + \Delta t} += \beta \Delta t^2 \Delta \Udd $\\
\>\>\> $ \Ud_{t + \Delta t} += \gamma \Delta t \Delta \Udd $\\
\>\> \} \\
\>\> theModel-$>$setResponse$(\U_{t + \Delta t}, \Ud_{t+\Delta t},
\Udd_{t+\Delta t})$ \\
\>\> theModel-$>$setUpdateDomain()
\end{tabbing}
Returns $0$ if successful. A warning message is printed and a negative number
returned if an error occurs: $-1$ if no associated AnalysisModel, $-2$
if the Vector objects have not been created, $-3$ if the Vector
objects and $\delta U$ are of different sizes. \\

{\em int sendSelf(int commitTag, Channel \&theChannel); } \\ 
Places in a Vector of size 6 the values of $\beta$, $\gamma$, {\em
dispFlag}, RayleighDampingFlag, $\alpha_M$ and $\beta_K$.  Then
invokes {\em sendVector()} on the Channel with this Vector. Returns
$0$ if successful, a warning message is printed and a $-1$ is 
returned if {\em theChannel} fails to send the Vector. \\ 

{\em int recvSelf(int commitTag, Channel \&theChannel, 
FEM\_ObjectBroker \&theBroker); } \\ 
Receives in a Vector of size 6 the values of $\beta$, $\gamma$, {\em
dispFlag}, RayleighDampingFlag, $\alpha_M$ and $\beta_K$. Returns $0$
if successful. A warning message is printed, $\gamma$ is set to 0.5,
$\beta$ to 0.25 and the Rayleigh damping flag set to {\em false}, and
a $-1$ is returned, if {\em theChannel} fails to receive the Vector.\\ 

{\em int Print(OPS_Stream \&s, int flag = 0);}\\
The object sends to $s$ its type, the current time, $\gamma$ and
$\beta$. If Rayleigh damping is specified, the constants $\alpha_M$ and
$\beta_K$ are also printed.






