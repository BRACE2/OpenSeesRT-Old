%File: ~/OOP/graph/graph/Graph.tex
%What: "@(#) Graph.tex, revA"

\noindent {\bf Files}   \\
\#include $<\tilde{ }$/graph/graph/Graph.h$>$  


\noindent {\bf Class Declaration}  \\
class Graph:  


\noindent {\bf Class Hierarchy} \\
{\bf Graph} 


\noindent {\bf Description}  \\
\indent Graph is a base class. A Graph is a container class
responsible for holding the vertex set and edge set. The class is
responsible for: \begin{enumerate} \item providing methods to add
vertices and edges. \item accessing the vertices and
edges. \end{enumerate} All the methods for the class are declared as
virtual to allow subclasses to be introduced. \\



\noindent {\bf Class Interface}  \\
\indent\indent // Constructors  \\
\indent\indent {\em Graph();}  \\
\indent\indent {\em Graph(int numVertices);} \\
\indent\indent {\em Graph(TaggedObjectStorage \&theVerticesStorage);} \\ \\
\indent\indent // Destructor  \\
\indent\indent {\em virtual~$\tilde{}$Graph();}  \\ \\
\indent\indent // Public Methods  \\
\indent\indent {\em virtual bool addVertex(Vertex *vertexPtr, bool
checkAdjacency = true);} \\
\indent\indent {\em virtual int addEdge(int vertexTag, int
otherVertexTag);  } \\ 
\indent\indent {\em virtual Vertex *getVertexPtr(int vertexTag);} \\
\indent\indent {\em virtual VertexIter \&getVertices(void);} \\
\indent\indent {\em virtual int getNumVertex(void) const;} \\
\indent\indent {\em virtual int getNumEdge(void) const;} \\
\indent\indent {\em virtual Vertex *removeVertex(int vertexTag, bool
removeEdgeFlag = true );} \\
\indent\indent {\em virtual void Print(OPS_Stream \&s, int flag = 0);} \\
\indent\indent {\em friend OPS_Stream \&operator<<(OPS_Stream \&s, const Graph 
\&G);} \\

\noindent {\bf Constructors}  \\
{\em Graph();}  

To create an empty Graph. Creates an ArrayOfTagged object of initial
size $32$ in which to store the Vertices. The ArrayOfTagged object is
used to store the Vertices.\\

{\em Graph(int numVertices);} 

To create an empty Graph. Creates an ArrayOfTagged object of initial
size {\em numVertices} in which to store the Vertices. The
ArrayOfTagged object is used to store the Vertices. \\ 

\indent {\em Graph(TaggedObjectStorage \&theVerticesStorage);}\\ 
To create an empty Graph. The {\em theVerticesStorage} object is used to
store the Vertices. {\em clearAll()} is invoked on this object to
ensure it is empty. \\

\noindent {\bf Destructor}  \\
{\em virtual~$\tilde{}$Graph();}  

Invokes {\em clearAll()} on the storage object used to store the
Vertices. It then invokes delete on the storage object used, which was
either passed or created in the constructor. \\

\noindent {\bf Public Methods }  \\
\indent {\em virtual bool addVertex(Vertex *vertexPtr, bool
checkAdjacency = true);} \\
Causes the graph to add a vertex to the graph. If {\em checkAdjacency}
is {\em true}, a check is made to ensure that all the Vertices in the
adjacency list of the Vertex are in the Graph. If a vertex in the
adjacency is not in the Graph the vertex is not added, a warning
message is printed and {\em false} is returned. If successful,
returns the result of invoking {\em addComponent()} on the
TaggedStorage object used to store the Vertices. \\



{\em virtual int addEdge(int vertexTag, int otherVertexTag); } \\
Causes the Graph to add an edge {\em (vertexTag,otherVertexTag)} to
the Graph. A check is first made to see if vertices with tags given by
{\em vertexTag} and {\em otherVertexTag} exist in the graph. If they
do not exist a $-1$ is returned, otherwise the method invokes {\em
addEdge()} on each of the corresponding vertices in the 
graph. Increments {\em numEdge} by $1$ and returns $0$ if sucessfull,
a $1$ if the edge already existed, and a $-2$ if one {\em addEdge()}
was successful, but the other was not.\\  

{\em virtual Vertex *getVertexPtr(int vertexTag);} \\
A method which returns a pointer to the vertex whose tag is given by {\em
vertexTag}. If no such vertex exists in the graph $0$ is
returned. Invokes {\em getComponentPtr(vertexTag)} on the vertex
storage object and casts this to a Vertex * if not null. \\

{\em virtual VertexIter \&getVertices(void);} \\
A method which returns a reference to the graphs VertexIter. This iter
can be used for iterating through the vertices of the graph. \\

{\em virtual int getNumVertex(void) const;} \\
A method to return the number of vertices in the graph. Invokes {\em
getNumComponents()} on the Vertex storage object.\\

{\em virtual int getNumEdge(void) const;} \\
A method to return the number of edges in the graph, returns {\em
numEdge}. \\ 

\indent {\em virtual Vertex *removeVertex(int vertexTag, bool
removeEdgeFlag = true );} \\
To remove the Vertex from the Graph whose tag is equal to {\em
vertexTag}. If {\em removeEdgeFlag} is {\em true} will also remove the
Vertex from the remaining Vertices adjacency lists. returns a pointer
to the removed Vertex if successful, $0$ if the Vertex was not in the
Graph. Invokes {\em removeComponent(vertexTag)} on the vertex
storage object and casts this to a Vertex * if not null. DOES NOT YET
DEAL WITH {\em removeEdgeFlag}. \\

{\em virtual void Print(OPS_Stream \&s, int flag =0);} \\
A method to print the graph. Invokes {\em Print(s, flag)} on the vertex
storage object. \\

\indent {\em friend OPS_Stream \&operator<<(OPS_Stream \&s, const Graph 
\&G);} \\
Invokes {\em Print()} on the Graph {\em G}.