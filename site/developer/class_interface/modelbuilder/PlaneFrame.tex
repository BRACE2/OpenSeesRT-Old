%File: ~/OOP/modelbuilder/PlaneFrame.tex
%What: "@(#) PlaneFrame.tex, revA"

\noindent {\bf Files}   \\
\#include $<$/modelbuilder/PlaneFrame.h$>$  


\noindent {\bf Class Declaration}  \\
class PlaneFrame: public ModelBuilder;  


\noindent {\bf Class Hierarchy} \\
ModelBuilder 

\indent\indent {\bf PlaneFrame} \\

\noindent {\bf Description} \\ 
\indent The PlaneFrame class is used to construct 2d plane frame
models from an input file of specified format. \\

\noindent {\bf Class Interface} \\ 
\indent // Constructor \\ 
{\em PlaneFrame(theDomain \&theDomain);}\\ 

// Destructor 

{\em virtual~ $\tilde{}$PlaneFrame();}\\ 

// Public Methods 

{\em virtual buildFE\_Model(void) = 0;} 



\noindent {\bf Constructor} \\ 
\indent {\em PlaneFrame(theDomain \&theDomain);}\\ 
The {\em theDomain} object is used by the ModelBuilder classes 
constructor. \\ 

\noindent {\bf Destructor} \\
\indent {\em virtual~ $\tilde{}$PlaneFrame();}\\ 
This does nothing. It is the responsibility of the Domain object to delete
all the domain components when the destructor is called on that object. \\

\noindent {\bf Public Methods} \\
{\em virtual buildFE\_Model(void) = 0;} 

The PlaneFrame will construct the Element, Node, Load and Constraint
objects and add them to the Domain object associated with the PlaneFrame. 
To do this the PlaneFrame object will prompt the user for the name of 
the input file; if the file cannot be opened an error message is
printed and the program terminates. A sample input file is given below: \\

\indent\indent 4 3 3 1 2\\
\indent\indent 1 3  0.0  0.0 \\
\indent\indent 2 3  0.0 10.0 \\
\indent\indent 3 3 20.0 10.0 \\
\indent\indent 4 3 20.0  0.0 \\
\indent\indent 2 1 10.0 200000.0 100.0 1 2 \\
\indent\indent 3 2 20.0 200000.0 100.0 2 3 \\
\indent\indent 2 3 10.0 200000.0 100.0 3 4 \\
\indent\indent 1 0 0.0 \\
\indent\indent 1 1 0.0 \\
\indent\indent 1 2 0.0 \\
\indent\indent 3 4 2 3 \\
\indent\indent 0 1 \\
\indent\indent 0 1 2 \\
\indent\indent 1.0 1.0 0.0 \\
\indent\indent 1.0 0.0 2.0 \\
\indent\indent 2 10.0 0.0 0.0 \\
\indent\indent 3 10.0 10.0 0.0 \\


\indent line 1 contains the number of nodes (4), elements (3), single
point constraints (3), multiple point constraints(1) and number of
nodal loads respectively (2). The next 4 lines contains the nodal
data; for each line a Node object is constructed using the tag and,
the number of dof, and the x and y coords specified. The next 3 lines
contains the elemental data; for each line a beam object is created,
the type of beam object depending on the first integer value of each
line (2 == beam2d02, 3 == beam2d03), each beam takes a tag, A,E,I and
the tags of the nodes at end1 and end2. 

The next 3 lines contains the single point
constraint data; for each line a SP\_Constraint object is created
(constrainedNode, constrainedDOF, value). 

The next 5 lines contain the data for the multipoint constraint, the
first line identifies the tag of the constrained node(3), the tag of
the retained node(4) and the number of dof in the relationship for
each node (2 for node 3 and 3 for node 4). The next line contains the
degrees-of-freedom in the constrained node $U_c$, (0 1), the next lines the
degrees-of-freedom in the retained node, $U_r$, (0 1 2) and the next
two lines the $C_{cr}$ matrix defining the relationship $U_c = C_{cr} U_r$.

whose tag is 5 which has two load cases. The next two lines contain
the NodalLoad information; for each line a NodalLoad is created which
acts on the node specified with the forces xForce, yForce and
moment. To returns 0 if successful, -1 otherwise. \\





