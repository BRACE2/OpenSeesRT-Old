%File: ~/OOP/convergenceTest/CTestNormUnbalance.tex
%What: "@(#) CTestNormUnbalance.tex, revA"

\noindent {\bf Files}   \\
\#include $<\tilde{}$/convergenceTest/CTestNormUnbalance.h$>$  


\noindent {\bf Class Declaration}  \\
class CTestNormUnbalance: public ConvergenceTest  


\noindent {\bf Class Hierarchy} \\
MovableObject 

\indent\indent ConvergenceTest \\
\indent\indent\indent {\bf CTestNormUnbalance} \\

\noindent {\bf Description}  \\
\indent A CTestNormUnbalance object is an object which can be used in an
algorithmic class to test if convergence has been achieved. The
CTestNormUnbalance class tests using the norm of the right hand side
Vector of a LinearSOE object and a user specified tolerance value. \\


\noindent {\bf Class Interface} \\
\indent\indent {// Constructors}  \\ 
\indent\indent {\em CTestNormUnbalance(double tol, int maxNumIter, int
printFlag);}  \\  
\indent\indent {\em CTestNormUnbalance();}  \\ \\
\indent\indent {// Destructor}  \\ 
\indent\indent {\em $\tilde{ }$CTestNormUnbalance();} \\ \\
\indent\indent {// Public Methods}  \\ 
\indent\indent {\em int setTolerance(double newTol);} \\
\indent\indent {\em int setEquiSolnAlgo(EquiSolnAlgo \&theAlgo);} \\
\indent\indent {\em int start(void);} \\
\indent\indent {\em int test(void);} \\
\indent\indent {\em int sendSelf(Channel \&theChannel, FEM\_ObjectBroker
\&theBroker);}\\ 
\indent\indent {\em int recvSelf(Channel \&theChannel, FEM\_ObjectBroker
\&theBroker);}\\ 

\noindent {\bf Constructors}  \\
\indent {\em CTestNormUnbalance(double tol, int maxNumIter, int printFlag);}  \\ 
{\em CTestNormUnbalance(double tol, int );}  

The integer {\em CLASS\_TAGS\_CTestNormUnbalance}, defined in
$<$classTags.h$>$, is passed to the ConvergenceTest constructor. 
Sets the tolerance used in {\em test()} to be {\em tol}, the max
number of iterations to be performed to {\em maxNumIter} and the print
flag used to determine what, if anything, is printed on each test to
{\em printFlag}. \\

{\em CTestNormUnbalance();}  

To be used by the FEM\_ObjectBroker object in parallel programs. The
integer {\em CLASS\_TAGS\_CTestNormUnbalance}, defined in
$<$classTags.h$>$, is passed to the ConvergenceTest constructor. Sets
the tolerance used in {\em test()} to be {\em 0.0} and {\em maxNumIter} 
to be $0$. These will be set when {\em recvSelf()} is invoked on the object. \\

\noindent {\bf Destructor} \\
\indent {\em virtual $\tilde{}$CTestNormUnbalance();} \\ 
Does nothing. \\

\noindent {\bf Public Methods }  \\
{\em int setTolerance(double newTol);} 

Sets the tolerance used in {\em test()} to be {\em newTol}. \\

{\em int setEquiSolnAlgo(EquiSolnAlgo \&theAlgo);} 

It sets a pointer to {\em theAlgo}'s LinearSOE object. Returns $0$ if
successful, a $-1$ is returned and an error message printed if no
LinearSOE object has been set in {\em theAlgo}. \\

{\em virtual void start(void);} 

Sets an integer indicating the current number of iterations, {\em
currentNumIter} to $1$. Returns $0$ if successfull, an error message
and $-1$ are returned if no LinearSOE object has been set. \\ 

{\em int test(void);} 

Returns {currentNumIter} if if the two norm of the LinearSOE objects B
Vector is less than the tolerance {\em tol}. If no LinearSOE has been
set $-2$ is returned. If the {\em currentNumIter} $>=$ {\em
maxNumIter} an error message is printed and $-2$ is returned. If none
of these conditions is met, the {\em currentnumIter} is incremented 
and $-1$ is returned. If the print flag is $0$ nothing is printed to
opserr during the method, if $1$ the current iteration and norm are
printed to opserr, and if $2$ the norm and number of iterations to convergence
are printed to opserr. \\ 


\indent {\em int sendSelf(int commitTag, Channel \&theChannel);}\\ 
Creates a Vector of size 2, puts the tolerance value {\em tol} and {\em numIncr}
in this, and then invokes {\em sendVector()} on {\em
theChannel}. Returns $0$ if successful. A warning message is printed
and a negative number if the Channel object fails to send the Vector.\\ 

\indent {\em int recvSelf(int commitTag, Channel \&theChannel,
FEM\_ObjectBroker \&theBroker);}\\ 
Creates a Vector of size 2, invokes {\em recvVector()} on {\em
theChannel}, and sets the values of {\em tol} and {\em numIncr}.
Returns $0$ if successful. If the Channel object fails to receive the
Vector, {\em tol} is set to $1.0e-8$, {\em numIter} to $25$, a warning
message is printed, and a negative number returned.
