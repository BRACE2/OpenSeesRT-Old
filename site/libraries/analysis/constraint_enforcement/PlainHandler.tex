%File: ~/OOP/analysis/handler/PlainHandler.tex
%What: "@(#) PlainHandler.tex, revA"

\noindent {\bf Files}   \\
\#include $<\tilde{ }$/analysis/handler/PlainHandler.h$>$  


\noindent {\bf Class Declaration}  \\
class PlainHandler: public ConstraintHandler  


\noindent {\bf Class Hierarchy} \\
MovableObject 

\indent\indent ConstraintHandler \\
\indent\indent\indent {\bf PlainHandler} \\

\noindent {\bf Description} \\ 
\indent The PlainHandler class is a class which only deals with
homogeneous single point constraints. To do this it creates regular
FE\_Element and DOF\_Group objects and enforces the constraints by
specifying that degrees-of-freedom which are constrained are not
assigned an equation number. Pointers to the DOF\_Group and
FE\_Element objects are kept in two arrays. \\

\noindent {\bf Class Interface} \\
// Constructor 

{\em PlainHandler();}\\  

// Destructor 

{\em $\tilde{ }$PlainHandler();}\\  

// Public Methods

\indent {\em int handle(const ID *nodeToBeNumberedLast
=0);} \\ 
{\em void clearAll(void);} \\ 

{\em int sendSelf(int commitTag, Channel \&theChannel); } 

\indent {\em int recvSelf(int commitTag, Channel \&theChannel,
FEM\_ObjectBroker \&theBroker); } \\

\noindent {\bf Constructor} \\
\indent {\em PlainHandler();}\\ 
The integer {\em HANDLER\_TAG\_PlainHandler} (defined in
$<$classTags.h$>$) is passed to the PlainHandler constructor. \\

\noindent {\bf Destructor} \\
\indent {\em virtual~ $\tilde{}$PlainHandler();}\\ 
Invokes the destructor on all the FE\_Element and DOF\_Group objects
created in {\em handle}. Then invokes the destructor on the two
pointer arrays. \\

\noindent {\bf Public Methods }\\
{\em int handle(const ID *nodeToBeNumberedLast =0);} 

Determines the number of FE\_Elements and DOF\_Groups needed from the
Domain (a one to one mapping between Elements and FE\_Elements and
Nodes and DOF\_Groups) Creates two arrays of pointers to store the
FE\_elements and DOF\_Groups, returning a warning message and a $-2$
or $-3$ if not enough memory is available for these arrays. Then the
object will iterate through the Nodes of the Domain, creating a
DOF\_Group for each node and setting the initial id for each dof to
$-2$ if no SP\_Constraint exists for the dof, or $-1$ if a
SP\_Constraint exists or $-3$ if the node identifier is in {\em
nodesToBeNumberedLast}. The object then iterates through the Elements
of the Domain creating a FE\_Element for each Element, if the Element
is a Subdomain {\em setFE\_ElementPtr()} is invoked on the Subdomain
with the new FE\_Element as the argument. If not enough memory is
available for any DOF\_Group or FE\_element a warning message is
printed and a $-4$ or $-5$ is returned. If any MP\_Constraint objects
exist in the Domain a warning message is printed and $-6$ is
returned. If all is successful, the method returns the number of
degrees-of-freedom associated with the DOF\_Groups in {\em
nodesToBeNumberedLast}. \\ 

{\em void clearAll(void) =0;} \\
Currently this invokes delete on all the FE\_element and DOF\_Group
objects created in {\em handle()} and the arrays used to store
pointers to these objects. FOR ANALYSIS INVOLVING DYNAMIC LOAD
BALANCING, RE-MESHING AND CONTACT THIS MUST CHANGE. \\

{\em int sendSelf(int commitTag, Channel \&theChannel); } \\
Returns $0$. \\

{\em int recvSelf(int commitTag, Channel \&theChannel, FEM\_ObjectBroker
\&theBroker); } \\
Returns $0$. 