%File: ~/OOP/analysis/integrator/WilsonTheta.tex
%What: "@(#) WilsonTheta.tex, revA"

\noindent {\bf Files}   \\
\#include $<\tilde{ }$/analysis/integrator/WilsonTheta.h$>$  


\noindent {\bf Class Declaration}  \\
class WilsonTheta: public TransientIntegrator  


\noindent {\bf Class Hierarchy} \\
MovableObject 

\indent\indent Integrator \\
\indent\indent\indent IncrementalIntegrator \\
\indent\indent\indent\indent TransientIntegrator \\
\indent\indent\indent\indent\indent {\bf WilsonTheta} \\

\noindent {\bf Description} \\ 
\indent WilsonTheta is a subclass of TransientIntegrator which implements
the Wilson$\Theta$ method. In the Wilson$\Theta$ method, to determine the 
velocities, accelerations and displacements at time $t + \theta \Delta
t$, $\theta \ge 1.37$, for $\U_{t+ \theta \Delta t}$ 

\[ \R (\U_{t + \theta \Delta t}) = \P(t + \theta \Delta t) -
\F_I(\Udd_{t+ \theta \Delta t}) 
- \F_R(\Ud_{t + \theta \Delta t},\U_{t + \theta \Delta t}) \]

\noindent where we use following functions to relate $\Ud_{t + \theta
\Delta t}$ and $\Udd_{t + \theta \Delta t}$ to $\U_{t + \theta \Delta
t}$ and the response quantities at time $t$:

\[
\dot \U_{t + \theta \Delta t} = \frac{3}{\theta \Delta t} \left(
\U_{t + \theta \Delta t} - \U_t \right)
 - 2 \dot \U_t + \frac{\theta \Delta t}{2} \ddot \U_t 
\]

\[
\ddot \U_{t + \theta \Delta t} = \frac{6}{\theta^2 \Delta t^2}
\left( \U_{t+\theta \Delta t} - \U_t \right)
 - \frac{6}{\theta \Delta t} \dot \U_t -2 \Udd_t
\]

\noindent which  results in the following for determining the responses at
$t + \theta \Delta t$ 

\[ \left[ \frac{6}{\theta^2 \Delta t^2} \M + \frac{3}{\theta \Delta t}
\C + \K \right] \Delta \U_{t + \theta \Delta t}^{(i)} = \P(t + \theta
\Delta t) - \F_I\left(\Udd_{t+\theta \Delta  t}^{(i-1)}\right) 
- \F_R\left(\Ud_{t + \theta \Delta t}^{(i-1)},\U_{t + \theta \Delta
t}^{(i-1)}\right) \]


\noindent The response quantities at time $t + \Delta t$ are then
determined using the following

\[
\Udd_{t + \Delta t} = \Udd_t + \frac{1}{\theta} \left( \Udd_{t +
\theta \Delta t} - \Udd_t \right)
\]

\[ \Ud_{t + \Delta t} = \Ud_t + \frac{\Delta t}{2}\left( \Udd_{t +
\Delta t} + \Udd_t \right) \]

\[ \U_{t + \Delta t} = \U_t + \Delta t\Ud_t + \frac{\Delta t^2}{6}\left(
\Udd_{t + \Delta t} + 2 \Udd_t \right) \]
\\

\pagebreak
\noindent {\bf Class Interface} \\
// Constructors 

\indent {\em WilsonTheta();}\\ 
\indent {\em WilsonTheta(double theta);}\\ 
{\em WilsonTheta(double theta, double alphaM, double betaK);}\\ 

// Destructor 

{\em virtual~ $\tilde{}$WilsonTheta();}\\ 

// Public Methods 

{\em int formEleTangent(FE\_Element *theEle);} 

{\em int formNodTangent(DOF\_Group *theDof);} 

{\em int domainChanged(void);}

{\em int newStep(double deltaT);}

\indent {\em int update(const Vector \&$\Delta U$);} \\ 
{\em int commit(void);}\\ 

// Public Methods for Output

\indent {\em int sendSelf(int commitTag, Channel \&theChannel);}\\ 
\indent {\em int recvSelf(int commitTag, Channel \&theChannel,
FEM\_ObjectBroker \&theBroker);}\\ 
{\em int Print(OPS_Stream \&s, int flag = 0);}


\noindent {\bf Constructors} \\
\indent {\em WilsonTheta();}\\ 
The integer INTEGRATOR\_TAGS\_WilsonTheta is passed to the TransientIntegrator
constructor. $\Theta$ is set to 0.0. This constructor should only be
invoked by an FEM\_ObjectBroker. \\



\indent {\em WilsonTheta(double theta);}\\ 
Sets $\Theta$ to {\em theta}, $\gamma$ to $(1.5 - \alpha)$ and $\beta$
to $0.25*\alpha^2$. addition, a flag is set indicating that Rayleigh
damping will not be used. \\ 

\indent {\em WilsonTheta(double theta);}\\ 
Sets $\Theta$ to {\em theta}, $\gamma$ to $(1.5 - \alpha)$ and $\beta$
to $0.25*\alpha^2$. In addition, a flag is set indicating that Rayleigh
damping will not be used. \\

\indent {\em WilsonTheta(double theta, double alphaM, double betaK);}\\ 
This constructor is invoked if Rayleigh damping is to be used, 
i.e. $\D = \alpha_M M + \beta_K K$. Sets $\Theta$ to {\em theta},
$\gamma$ to $(1.5 - \alpha)$, $\beta$ to $0.25*\alpha^2$, $\alpha_M$ to
{\em alphaM} and $\beta_K$ to {\em betaK}. Sets a flag indicating whether the
incremental solution is done in terms of displacement or acceleration
to {\em dispFlag} and a flag indicating that Rayleigh damping will 
be used. \\ 


\noindent {\bf Destructor} \\
\indent {\em virtual~ $\tilde{}$WilsonTheta();}\\ 
Invokes the destructor on the Vector objects created. \\

\noindent {\bf Public Methods}\\
{\em int formEleTangent(FE\_Element *theEle);} 

This tangent for each FE\_Element is defined to be $\K_e = c1 \K
+ c2  \D + c3 \M$, where c1,c2 and c3 were determined in the last invocation
of the {\em newStep()} method. Returns $0$ after performing the
following operations:  
\begin{tabbing}
while \= \+ while \= while \= \kill
if (RayleighDamping == false) \{ \+ \\
theEle-$>$zeroTang() \\
theEle-$>$addKtoTang(c1) \\
theEle-$>$addCtoTang(c2) \\
theEle-$>$addMtoTang(c3)  \- \\
\} else \{ \+ \\
theEle-$>$zeroTang() \\
theEle-$>$addKtoTang(c1 + c2 * $\beta_K$) \\
theEle-$>$addMtoTang(c3 + c2 * $\alpha_M$)  \- \\ 
\}
\end{tabbing}

{\em int formNodTangent(DOF\_Group *theDof);} \\
This performs the following:
\begin{tabbing}
while \= \+ while \= while \= \kill
theDof-$>$zeroUnbalance() \\
if (RayleighDamping == false)  \+ \\
theDof-$>$addMtoTang(c3)  \- \\
else \+ \\
theDof-$>$addMtoTang(c3 + c2 * $\alpha_M$)  \- \\ 
\end{tabbing}


{\em int domainChanged(void);}\\
If the size of the LinearSOE has changed, the object deletes any old Vectors
created and then creates $6$ new Vector objects of size equal to {\em
theLinearSOE-$>$getNumEqn()}. There is a Vector object created to store
the current displacement, velocity and accelerations at times $t$ and
$t + \Delta t$ (between {\em newStep()} and {\em commit()} the $t +
\Delta t$ quantities store $t + \Theta \Delta t$ quantities).
The response quantities at time $t + \Delta t$ are
then set by iterating over the DOF\_Group objects in the model and
obtaining their committed values. 
Returns $0$ if successful, otherwise a warning message and a negative
number is returned: $-1$ if no memory was available for constructing
the Vectors. \\


{\em int newStep(double $\Delta t$);}\\
The following are performed when this method is invoked:
\begin{enumerate}
\item First sets the values of the three constants {\em c1}, {\em c2}
and {\em c3}: {\em c1} is set to $1.0$, 
{\em c2} to $3 / (\Theta
\Delta t)$ and {\em c3} to $6 / (\Theta \Delta t)^2)$. 
\item Then the Vectors for response quantities at time $t$ are set
equal to those at time $t + \Delta t$.
\begin{tabbing}
while \= while \= while \= while \= \kill
\>\> $ \U_t = \U_{t + \Delta t}$ \\
\>\> $ \Ud_t = \Ud_{t + \Delta t} $ \\
\>\> $ \Udd_t = \Udd_{t + \Delta t} $ 
\end{tabbing}
\item Then the velocity and accelerations approximations
at time $t + \Theta \Delta t$ are set using the difference
approximations,
\begin{tabbing}
while \= while \= while \= while \= \kill
\>\> $ \U_{t + \theta \Delta t} = \U_t $ \\
\>\> $ \dot \U_{t + \theta \Delta t} = - 2 \dot \U_t + \frac{\theta
\Delta t}{2} \ddot \U_t  $\\
\>\> $ \ddot \U_{t + \theta \Delta t} = - \frac{6}{\theta \Delta t}
\dot \U_t -2 \Udd_t $ 
\end{tabbing}
\item The response quantities at the DOF\_Group objects are updated
with the new approximations by invoking {\em setResponse()} on the
AnalysisModel with quantities at time $t + \Theta \Delta t$.
\begin{tabbing}
while \= while \= while \= while \= \kill
\>\> theModel-$>$setResponse$(\U_{t + \theta \Delta t}, \Ud_{t+\theta
\Delta t}, \Udd_{t+ \theta \Delta t})$ 
\end{tabbing}
\item current time is obtained from the AnalysisModel, incremented by
$\Theta \Delta t$, and {\em applyLoad(time, 1.0)} is invoked on the
AnalysisModel. 
\item Finally {\em updateDomain()} is invoked on the AnalysisModel.
\end{enumerate}
The method returns $0$ if successful, otherwise a negative number is
returned: $-1$ if $\gamma$ or $\beta$ are $0$, $-2$ if {\em dispFlag}
was true and $\Delta t$ is $0$, and $-3$ if {\em domainChanged()}
failed or has not been called. \\



{\em int update(const Vector \&$\Delta U$);} \\
Invoked this first causes the object to increment the DOF\_Group
response quantities at time $t + \Theta \Delta t$. The displacement Vector is  
incremented by $ c1 * \Delta U$, the velocity Vector by $
c2 * \Delta U$, and the acceleration Vector by $c3 * \Delta U$. 
The response quantities at the DOF\_Group objects are then updated
with the new approximations by invoking {\em setResponse()} on the
AnalysisModel with displacements, velocities and accelerations at time
$t + \Theta \Delta t$.
Finally {\em updateDomain()} is invoked on the AnalysisModel. 
\begin{tabbing}
while \= while \= while \= while \= \kill
\>\> $ \U_{t + \theta \Delta t} += \Delta \U$ \\
\>\> $ \dot \U_{t + \theta \Delta t} += \frac{3}{\theta \Delta t}
\Delta \U  $\\
\>\> $ \ddot \U_{t + \theta \Delta t} += \frac{6}{\theta^2 \Delta
t^2} \Delta \U $ \\ 
\>\> theModel-$>$setResponse$(\U_{t + \alpha \theta t}, \Ud_{t+\theta
\Delta t}, \Udd_{t+ \theta \Delta t})$ \\
\>\> theModel-$>$updateDomain()
\end{tabbing}
Returns $0$ if successful. A warning message is printed and a negative number
returned if an error occurs: $-1$ if no associated AnalysisModel, $-2$
if the Vector objects have not been created, $-3$ if the Vector
objects and $\Delta U$ are of different sizes. \\



{\em int commit(void);}\\
First the quantities at time $t + \Delta t$ are determined using the
quantities at time $t$ and $t + \Theta \Delta t$.
Then the response quantities at the DOF\_Group objects are updated
with the new approximations by invoking {\em setResponse()} on the
AnalysisModel with displacement, velocity and accelerations at time $t +
\Delta t$. The time is obtained from the AnalysisModel and $(\Theta
-1) \Delta t$ is subtracted from the value. The time is set in the
Domain by invoking {\em setCurrentDomainTime(time)} on the
AnalysisModel. Finally {\em updateDomain()} and {\em commitDomain()}
are invoked on the AnalysisModel. 
\begin{tabbing}
while \= while \= while \= while \= \kill
\>\> $\Udd_{t + \Delta t} = \Udd_t + \frac{1}{\theta} \left( \Udd_{t +
\theta \Delta t} - \Udd_t \right)$ \\
\>\> $ \Ud_{t + \Delta t} = \Ud_t + \frac{\Delta t}{2}\left( \Udd_{t +
\Delta t} + \Udd_t \right) $ \\
\>\> $ \U_{t + \Delta t} = \U_t + \Delta t\Ud_t + \frac{\Delta t^2}{6}\left(
\Udd_{t + \Delta t} + 2 \Udd_t \right) $ \\
\>\> theModel-$>$setResponse$(\U_{t + \Delta t}, \Ud_{t+
\Delta t}, \Udd_{t+\Delta t})$ \\
\>\> time = theModel-$>$getDomainTime() \\
\>\> time -= $(\theta -1) * \Delta t$ \\
\>\> theModel-$>$setTime(time) \\
\>\> theModel-$>$commitDomain()
\end{tabbing}
Returns $0$ if successful, a warning
message and a negative number if not: $-1$ if no AnalysisModel
associated with the object and $-2$ if {\em commitDomain()} failed. \\


{\em int sendSelf(int commitTag, Channel \&theChannel); } \\ 
Places $\Theta$, rayleigh damping flag, $\alpha_M$ and $\beta_K$ in a
vector if size 4 and invokes {\em sendVector} on the Channel with this
Vector. Returns $0$ if successful, a warning message is printed and a
$-1$ is returned if {\em theChannel} fails to send the Vector. \\ 

{\em int recvSelf(int commitTag, Channel \&theChannel, 
FEM\_ObjectBroker \&theBroker); } \\ 
Receives in a Vector of size 4 the value of $\Theta$, the rayleigh
damping flag, $\alpha_M$ and $\beta_K$.. Returns $0$ if
successful, a warning message is printed,  $\Theta$ is set to $0$, the
rayleigh damping flag to {\em false}, and a $-1$ is returned if {\em
theChannel} fails to receive the Vector.\\ 

{\em int Print(OPS_Stream \&s, int flag = 0);}\\
The object sends to $s$ its type, the current time, $\alpha$, $\gamma$ and
$\beta$. 

