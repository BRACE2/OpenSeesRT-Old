%File: ~/OOP/element/ElementalLoad.tex
%What: "@(#) ElementalLoad.tex, revA"

\noindent {\bf Files}   \\
\#include $<\tilde{ }$/element/ElementalLoad.h$>$  


\noindent {\bf Class Declaration}  \\
class ElementalLoad: public Load 


\noindent {\bf Class Hierarchy} \\
TaggedObject 

MovableObject 

\indent\indent DomainComponent \\
\indent\indent\indent Load \\
\indent\indent\indent\indent {\bf ElementalLoad} \\

\noindent {\bf Description}  \\
\indent ElementalLoad is an abstract class, i.e. no instances of
ElementalLoad will exist. The ElementalLoad class provides the
interface that all ElementalLoad writers must provide when
introducing new ElementalLoad classes. \\ 

\noindent {\bf Class Interface} \\
\indent\indent // Constructors \\
\indent\indent {\em ElementalLoad(int elementTag, int tag, int classTag);}  \\ 
\indent\indent {\em ElementalLoad(int classTag);}  \\ \\
\indent\indent // Destructor \\
\indent\indent {\em virtual~ $\tilde{}$ElementalLoad();}\\ \\
\indent\indent // Public Methods  \\
\indent\indent {\em virtual int getElementTag(void) const;} \\



\noindent {\bf Constructor}  \\
{\em ElementalLoad(int elementTag, int tag, int classTag);}  

Provided to allow subclasses to construct an ElementalLoad object
associated with the Element whose unique identifier in the Domain will
be {\em elementTag}. The integers {\em tag} and and {\em classTags}
are passed to the Load constructor. \\ 

{\em ElementalLoad(int classTag);}  

Provided so that a FEM\_ObjectBroker can construct an object. $0$ and
{\em classTag} are passed to the Load classes constructor. The data
for the object is filled in when {\em recvSelf()} is invoked on the
object.\\

\noindent {\bf Destructor} \\
\indent {\em virtual~ $\tilde{}$ElementalLoad();}\\ 
Does nothing. Provided so that the ElementalLoad subclasses destructor
will be called. \\

\noindent {\bf Public Methods }  \\
\indent\indent {\em virtual int getElementTag(void) const;} \\
Returns the integer {\em elementTag} passed in the constructor. 
