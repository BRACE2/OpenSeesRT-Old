%File: ~/OOP/matrix/Matrix.tex
%What: "@(#) Matrix.tex, revA"

\noindent {\bf Files}   \\
\#include $<\tilde{}$/matrix/Matrix.h$>$  


\noindent {\bf Class Declaration}  \\
class Matrix:  


\noindent {\bf Class Hierarchy} \\
{\bf Matrix} 


\noindent {\bf Description}  \\
\indent The Matrix class provides the matrix abstraction. A matrix of
order numRows X numCols is an ordered 2d array of numbers arranged in
numRows rows and numCols columns. For example, a matrix $A$ of order 2
X 3:
$$ A =
\left[
\begin{array}{ccc}
a_{0,0} & a_{0,1}  & a_{0,2}  \\
a_{1,0} & a_{1,1} & a_{1,2}  \\
\end{array}
\right] 

$$

$$


The Matrix class provides the implementation of a general unsymmetric
matrix. The data for the matrix is stored in a 1d double array of size
numRows*numCols with the data for $a_{i,j}$ located at j*numRows + i
in the 1d array. This is similar to the ordering of a Fortran 2d array
and will permit calls to numerical Fortran libraries, e.g. BLAS, for
certain method calls at a future stage. At present no subclassing is
permitted (THIS MAY CHANGE), the reason for this is that the Matrix
objects are envisioned to be small scale matrices primarily used for
the passing of data between objects in the system. To allow subclassing
could reduce the efficiency of the program due to the manner in which 
virtual functions are implemented.\\ 

\noindent {\bf Class Interface} \\
\indent {// Constructors}  \\ 
{\em  Matrix();}  

{\em  Matrix(int nrows, int ncols);}  

{\em  Matrix(double *data, int nRows, int nCols)}  

{\em  Matrix(const Matrix \&M); }  \\ 

\indent {// Destructor}  \\ 
{\em  $\tilde{}$Matrix();} \\ 

\indent {// Public Methods}  \\ 
{\em  inline int noRows() const;} 

{\em  inline int noCols() const;} 

{\em  void Zero();} 

\indent {\em  int Assemble(const Matrix \&M,const ID \&rows,
const ID \&cols,  double fact = 1.0);} \\
{\em  int Solve(const Vector \&b, Vector \&x);} 

{\em  int Solve(const Matrix \&b, G3MAtrix \&x);} 

{\em  Vector Solve(const Vector \&V);} 

{\em  int addMatrix(double factThis, const Matrix \&other, double factOther);}

{\em  int addMatrixProduct(double factThis, const Matrix \&A, const Matrix \&B, double factOther);}

{\em  int addMatrixTripleProduct(double factThis, const Matrix \&A, const Matrix \&B, double factOther);}

{\em  int addMatrixTripleProduct(double factThis, const Matrix \&T, const Matrix \&B, double factOther);}

\indent {\em  int addMatrixTripleProduct(double factThis, const Matrix \&A, const Matrix \&B, const Matrix \&C, double factOther);}\\\\ 
\indent {// Overloaded Operator Functions}  \\ 
{\em  inline double \&operator()(int row, int col);} 

\indent {\em  inline Matrix operator()(const ID \&rows, const
ID \&cols) const;}\\  
{\em  Matrix \&operator=(const Matrix \&M);}

{\em  Matrix \&operator+=(double fact);} 

{\em  Matrix \&operator-=(double fact);} 

{\em  Matrix \&operator*=(double fact);} 

{\em Matrix \&operator/=(double fact); } 

{\em Matrix operator+(double fact) const;} 

{\em Matrix operator-(double fact) const;} 

{\em Matrix operator*(double fact) const;} 

{\em Matrix operator/(double fact) const;} 

{\em Vector operator*(const Vector \&V) const;} 

\indent {\em Vector operator{ }$\hat{ }$(const
Vector \&V) const; } \\ 
{\em Matrix operator+(const Matrix \&M) const;} 

{\em Matrix operator-(const Matrix \&M) const;} 

{\em Matrix operator*(const Matrix \&M) const;} 

\indent {\em Matrix operator{ }$\hat{ }$(const Matrix \&M) const;} \\ 
{\em Matrix \&operator+=(const Matrix \&M); } 

{\em Matrix \&operator-=(const Matrix \&M); } 

{\em void Output(OPS_Stream \&s) const;} 

{\em void Input(istream \&s);} 

\indent {\em friend OPS_Stream \&operator$<<$(OPS_Stream \&s, const
Matrix \&M);} \\ 
\indent {\em friend istream \&operator$>>$(istream \&s, const
Matrix \&M);} \\ 


\noindent {\bf Constructors}  \\
{\em Matrix();}  

To construct a Matrix with numRows = $0$, and numCols = $0$. No memory
is allocated to store the data.\\ 

{\em  Matrix(int nrows, int ncols);}  

To construct a Matrix with numRows = {\em nrows}, and numCols =
{\em ncols} and all coefficients equal to $0$. Allocates memory for
the storage of the data. If numRows $<$ $0$, numCols $<$ $0$ or not
enough memory is available available, an error message is printed and
a Matrix with numRows = $0$ and numCols = $0$ is returned. Before the
Matrix is returned, {\em Zero()} is called on the Matrix.\\ 

{\em  Matrix(double *data, int nrows, int ncols);}  

To construct a Matrix with numRows = {\em nrows}, and numCols =
{\em ncols} and all coefficients equal to $0$. The memory for storage
of the data is found at the location pointed to by {\em data}. Note
that this memory must have been previously allocated and it must be of
appropriate size, erroneous results and segmentation faults can occur
otherwise. No additional memory is allocated for the storage of the
data. \\ 

{\em Matrix(const Matrix \&M); }  

To construct a Matrix using another Matrix. The new Matrix will be a
general matrix that is identical to the Matrix {\em M}, i.e. same size
and identical components. If not enough memory is available, an error
message is printed and a Matrix with numRows = $0$ and numCols = $0$
is returned. The constructor tests for the type of $M$ to see whether
the performance can be improved, by avoiding having to call $M$'s
overloaded (i,j) operators if $M$ is of type genMatrix. \\ 

\noindent {\bf Destructor} \\
\indent {\em  $\tilde{}$Matrix();} \\ 
Free's up any memory allocated in the constructor. Note that if the
third constructor had been invoked, the memory passed is not released
back to the system (this must be done later by the user of this
constructor). \\

\noindent {\bf Public Methods }  \\
{\em  inline int noRows() const;} 

Returns the number of rows, numRows, of the Matrix. The method is
declared inline for the compiler to produce faster code which does not
require a method call. \\

{\em  inline int noCols() const;} 

Returns the number of columns, numCols, of the Matrix. \\

{\em  void Zero();} 

Zero's out the Matrix, i.e. sets all the components of the matrix to
$0$. The method tests for the type of Matrix, to see whether the
performance can be improved by avoiding having to call the overloaded
(i,j) operators if the Matrix is of type genMatrix. \\ 

\indent {\em  Assemble(const Matrix \&M, 
const ID \&rows, const ID \&cols,
double fact = 1.0);} \\
Assembles into the current Matrix the Matrix {\em M}. The contents of the
current matrix at location ({\em rows(i),cols(j)}) is set equal to the
current value plus {\em fact} times the value of the matrix {\em M}
at location ({\em i,j }). A warning is printed for every {\em rows(i),
cols(j)} specified which is outside the bounds of the matrix. \\

{\em  Vector Solve(const Vector \&V);} 

Will solve the equation {\em $Ax=V$} for the Vector {\em x}, which is
returned. At the moment the current matrix is assumed to be symmetric
positive definite. THIS IS TO CHANGE. A Vector is created using
{\em V}. If this Vector is not of the correct size or if an error occurs
during factorization a warning message is printed and the Vector {\em x}
is returned. \\ 

\indent {\em  void addMatrix(const Matrix \&other, double fact
= 1.0);} \\ 
To add a factor {\em fact} times the Matrix {\em other} to the current
Matrix. The size of the other Matrix is first checked to see sizes are 
compatible; if size are not compatible nothing is done and a warning
message is printed. The method tests for the type of {\em other}, to see
whether the performance can be improved by avoiding having to call
{\em other}'s overloaded (i,j) operators, if {\em other} is of type
genMatrix. \\ 


\noindent {\bf Overloaded Operator Functions}  \\
{\em  inline double \&operator()(int row, int col) const;} 

Returns the data at location({\em row,col}) in the Matrix. Assumes
({\em row,col}) is a valid location in the Matrix, a segmentation
fault or erroneous results can occur if this is not the case.\\

{\em  inline double \&operator()(int row, int col);} 

Used to set the data at location({\em row,col}) in the Matrix. Assumes
({\em row,col}) is a valid location in the Matrix, a segmentation
fault or erroneous results can occur if this is not the case. \\

\indent {\em  Matrix operator()(const ID \&rows, const ID \&
cols) const;}\\ 
Returns a new Matrix of dimension ({\em rows.Size()}, {\em
cols.Size()}). The contents of the new matrix are given by the
contents of the current matrix at the locations given by the {\em
rows} and {\em cols} objects. For example the contents of the new
matrix at location ({\em i,j}) are equal to the contents of the
current matrix at location ({\em rows(i),cols(j)}). Assumes ({\em
row,col}) is a valid location in the Matrix, a segmentation fault or
erroneous results can occur if this is not the case. \\

{\em  Matrix \&operator=(const Matrix \&M);}

Sets the current Matrix to be equal to the Matrix given by {\em M}. If
the Matrices are of different sizes, the current data is deallocated
and additional space allocated before the contents are copied. If not
enough space can be allocated for the new data, an error message is
printed and a Matrix of size $0$ x $0$ is returned. The method tests
for the type of {\em M}, to see whether the performance can be
improved by avoiding having to call $M$'s overloaded (i,j) operators,
if $M$ is of type genMatrix. This method must be implemented by each
subclass. \\ 

{\em  Matrix \&operator+=(double fact);} 

A method to add {\em fact} to each component of the current Matrix. 
The method tests for the type of the current Matrix, to see whether
the performance can be improved by avoiding having to call the
overloaded (i,j) operators, if the current Matrix is of type
genMatrix. \\ 

{\em Matrix \&operator-=(double fact);} 

A method to subtract {\em fact} from each component of the current Matrix. 
The method tests for the type of the current Matrix, to see whether
the performance can be improved by avoiding having to call the
overloaded (i,j) operators, if the current Matrix is of type
genMatrix. \\ 

{\em Matrix \&operator*=(double fact);} 

A method to multiply each component of the current Matrix by {\em fact}. 
The method tests for the type of the current Matrix, to see whether
the performance can be improved by avoiding having to call the
overloaded (i,j) operators, if the current Matrix is of type
genMatrix. \\ 

{\em Matrix \&operator/=(double fact); } 

A method which will divide each component of the current Matrix
by {\em fact}. If {\em fact} is equal to zero, an error message is
printed and the contents of the Matrix are set to
MATRIX\_VERY\_LARGE\_VALUE (defined in $<$Matrix.h$>$). The method
tests for the type of the current Matrix, to see whether the
performance can be improved by avoiding having to call the overloaded
(i,j) operators, if the current Matrix is of type genMatrix. \\ 

{\em Matrix operator+(double fact) const;} 

A method to return a new Matrix, whose components are equal to the
components of the current Matrix plus the value {\em fact}. A new
Matrix object is constructed, using the current Matrix as the
argument to the constructor. The {\em +=} operator is then invoked 
on this Matrix with {\em fact} as the argument, and the new Matrix is
then returned. \\ 

{\em Matrix operator-(double fact) const;} 

A method to return a new Matrix, whose components are equal to the
components of the current Matrix minus the value {\em fact}. A new
Matrix object is constructed, using the current Matrix as the
argument to the constructor. The {\em -=} operator is then invoked 
on this Matrix with {\em fact} as the argument, and this new Matrix is
then returned. \\ 


{\em Matrix operator*(double fact) const;} 

A method to return a new Matrix, whose components are equal to the
components of the current Matrix times the value {\em fact}. A new
Matrix object is constructed, using the current Matrix as the
argument to the constructor. The {\em *=} operator is then invoked 
on this Matrix with {\em fact} as the argument, and this new Matrix is
then returned. \\ 

{\em Matrix operator/(double fact) const;} 

A method to return a new Matrix whose components are equal to the
components of the current Matrix divided the value {\em fact}. A new
Matrix object is constructed by using the current Matrix as the
argument to the constructor. The {\em /=} operator is then invoked 
on this Matrix with {\em fact} as the argument, and this new Matrix is
then returned. \\ 

{\em Vector operator*(const Vector \&V) const;} 

A method to return a new Vector, of size numRows, whose components are
equal to the product of the current Matrix times the Vector {\em
V}. If the current Matrix and Vector {\em V} are not compatible,
i.e. V.Size() is not equal to numCols, an error message is printed and 
a zero Vector of size equal to the number of rows in the current
Matrix is returned. The method tests for the type of the current
Matrix, to see whether the performance can be improved by avoiding
having to call the overloaded (i,j) operators, if the current Matrix
is of type genMatrix. \\ 

\indent {\em Vector operator { }$ \hat{ }$(const Vector
\&V) const; } \\ 
A method to return a new Vector, of size numCols, whose components are
equal to the product of the transpose of the current Matrix times the
Vector {\em V}. If the current Matrix and Vector {\em V} are not
compatible, i.e. V.Size() is not equal to numRows, an error message is
printed and a zero Vector of size equal to the number of columns in
the current Matrix is returned. The method tests for the type of the
current Matrix, to see whether the performance can be improved by
avoiding having to call the overloaded (i,j) operators, if the current
Matrix is of type genMatrix. \\ 

{\em Matrix operator+(const Matrix \&M) const;} 

A method to return a new Matrix equal to the sum of the current Matrix
and the Matrix {\em M}. It does this by creating a new matrix passing
itself as an argument to the constructor. The {\em addMatrix()} method
is then invoked on this new Matrix with $M$ and $-1$ as the
arguments. The new Matrix is then returned. \\ 

{\em Matrix operator-(const Matrix \&M) const;} 

A method to return a new Matrix equal to the the current Matrix minus
the Matrix {\em M}. It does this by creating a new matrix passing
itself as an argument to the constructor. The {\em addMatrix()} method
is then invoked on this new Matrix with $M$ and $-1$ as the
arguments. The new Matrix is then returned. \\ 

{\em Matrix operator*(const Matrix \&M) const;} 

A method to return a new Matrix equal to the product of the current
Matrix and the Matrix {\em M}. It does this by first creating a new
Matrix of size numRows and M.numCols. The contents of this new Matrix
are then determined and the resulting Matrix is returned. If the two
matrices are of incompatible sizes, a warning message is printed and a
zeroed Matrix is returned. The method tests for the type of the
current Matrix, to see whether the performance can be improved by
avoiding having to call the overloaded (i,j) operators, if {\em M} 
is of type genMatrix. \\

\indent {\em Matrix operator{ }$ \hat{ }$(const Matrix \&M)
const;} \\ 
A method to return a new Matrix equal to the product of the transpose
of the current Matrix and the Matrix {\em M}. It does this by first
creating a new Matrix of size numRows and M.numRows. The contents of
this new Matrix are then determined and the resulting Matrix is
returned. If the two matrices are of incompatible sizes, a warning
message is printed and a zeroed Matrix is returned. The method tests
for the type of the current Matrix, to see whether the performance can
be improved by avoiding having to call the overloaded (i,j) operators,
if {\em M} is of type genMatrix. \\

{\em void Output(OPS_Stream \&s) const;} 

To print the contents of the Matrix to the output stream {\em s}. The
method will print the contents one row at a time. \\ 

{\em void Input(istream \&s);} 

To read the contents of the Matrix from the input stream {\em s}. The method expects the components one row at a time. \\

{\em friend OPS_Stream \&operator$<<$(OPS_Stream \&s, const Matrix \&M);} \\
A function to print out the contents of the Matrix {\em M} to the
output stream {\em s}. Does this by invoking {\em Output()} on the
Matrix {\em M}.\\ 

{\em friend OPS_Stream \&operator$>>$(istream \&s, const Matrix \&M);} \\
A function to print out the contents of the Matrix {\em M} to the
output stream {\em s}. Does this by invoking {\em Output()} on the
Matrix {\em M}.\\ 

