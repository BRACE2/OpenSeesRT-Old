%File: ~/OOP/analysis/handler/ConstraintHandler.tex
%What: "@(#) ConstraintHandler.tex, revA"

\noindent {\bf Files}   \\
\#include $<\tilde{ }$/analysis/handler/ConstraintHandler.h$>$  


\noindent {\bf Class Declaration}  \\
class ConstraintHandler: public MovableObject  


\noindent {\bf Class Hierarchy} \\
MovableObject 

\indent\indent {\bf ConstraintHandler} \\

\noindent {\bf Description} \\ 
\indent The ConstraintHandler class is an abstract base class. Its purpose is
to define the interface common among all subclasses.  A constraint
handler is responsible for: \begin{enumerate} \item creating the
FE\_Element and DOF\_Group objects and adding them to the
AnalysisModel. \item setting the initial dof equation numbers to $-1$,
$-2$ or $-3$. A $-1$ indicates to the DOF\_Numberer object that no
equation number is to be allocated for this dof, a $-3$ that this dof
is to be among the last group of dof to be numbered. \item deleting
the DOF\_Group and FE\_Element objects that it created.\end{enumerate}


\noindent {\bf Class Interface} \\
// Constructor 

{\em ConstraintHandler(int classTag);}\\  

// Destructor 

{\em virtual~ $\tilde{}$ConstraintHandler();}\\  

// Public Methods

{\em virtual void setLinks(Domain \&theDomain, 

\indent\indent\indent\indent AnalysisModel \&theModel, \\
\indent\indent\indent\indent Integrator \&theIntegrator);} \\
\indent {\em virtual int handle(const ID *nodeToBeNumberedLast
=0) =0;} \\ 
{\em virtual void clearAll(void) =0;} \\ 

// Protected Methods  

{\em Domain *getDomainPtr(void) const;} 

{\em AnalysisModel *getAnalysisModelPtr(void) const;} 

{\em Integrator *getIntegratorPtr(void) const;} 



\noindent {\bf Constructor} \\
\indent {\em ConstraintHandler(int classTag);}\\ 
The integer {\em classTag} is passed to the MovableObject constructor. \\

\noindent {\bf Destructor} \\
\indent {\em virtual~ $\tilde{}$ConstraintHandler();}\\ 
Does nothing. \\

\noindent {\bf Public Methods }\\
{\em virtual void setLinks(Domain \&theDomain, 

\indent\indent\indent\indent\indent AnalysisModel \&theModel, \\
\indent\indent\indent\indent\indent Integrator \&theIntegrator);} \\
Invoked to set the links that the ConstraintHandler will need. These
include links to the Domain, {\em theDomain}, for which the
ConstraintHandler object will apply the constraints and the
AnalysisModel, {\em theModel}, to which the ConstraintHandler will add
the FE\_Element and DOF\_Group objects. \\

{\em virtual int handle(const ID *nodesToBeNumberedLast =0) =0;} 

Invoked to handle the constraints imposed on the domain by the
SP\_Constraints and MP\_Constraints. The ConstraintHandler object does
this by instantiating the appropriate FE\_ELement and DOF\_Group objects 
and adding them to the AnalysisModel. For all the dofs in each
DOF\_Group the ConstraintHandler sets initial equation numbers as
either $-1$, $-2$ or $-3$: A $-1$ indicates to the DOF\_Numberer object
that no equation number is to be allocated for this dof, a $-2$ that
an equation number is to be given for the dof, and a $-3$ that an
equation number is to be allocated and that this dof is to
be among the last group of dof to be numbered,i.e. all dof initially
assigned a $-3$ are to be given a higher equation number than those
given a $-2$. Those dof with a $-3$ should include all those dof
associated with the nodes whose tags are in {\em
nodesToBeNumberedLast}. Returns a positive number if successfully, a
negative integer if not; the positive number is to be set at the
number of dof assigned a value $-3$ (this will be the number of
external dof for a subdomain), the negative value of which depends on
the type of ConstraintHandler. For subdomains the constraint handler
is responsible for setting the FE\_Element by calling {\em
setFE\_elementPtr}. \\    

{\em virtual void clearAll(void) =0;} \\
Invoked to inform the ConstraintHandler object that the FE\_Elements
and DOF\_Groups it constructed are no longer part of the AnalysisModel. The
ConstraintHandler can delete these objects if necessary; or the
ConstraintHandler can store them and use them in subsequent calls to
{\em handle()}. \\ 

\noindent {\bf Protected Methods}  \\
{\em Domain *getDomainPtr(void) const;} 

A const member function to return the Domain object associated with
the ConstraintHandler, {\em theDomain}. \\

{\em AnalysisModel *getAnalysisModelPtr(void) const;} \\
A const member function to return the AnalysisModel object associated with
the ConstraintHandler, {\em theModel}. \\

{\em Integrator *getIntegratorPtr(void) const;} \\
A const member function to return the Integrator object associated with
the ConstraintHandler, {\em theIntegrator}. 





